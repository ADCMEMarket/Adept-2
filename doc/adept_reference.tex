\documentclass[10pt,a4,landscape]{article}
% Page set up
\setlength{\oddsidemargin}{0cm} %{0.5cm}
\setlength{\evensidemargin}{0cm} %{0.5cm}
\setlength{\topmargin}{-5cm}
\setlength{\textheight}{24cm}
\setlength{\textwidth}{16cm}
\setlength{\marginparsep}{0.5cm}
\setlength{\marginparwidth}{0cm}
\setlength{\parindent}{1em}
\setlength{\parskip}{0cm}
\renewcommand{\baselinestretch}{1.1}
\sloppy

\usepackage{lmodern}\usepackage[T1]{fontenc}
\usepackage{color}
%\renewcommand{\rmdefault}{sbc}
\usepackage[figuresright]{rotating}
\DeclareFontFamily{T1}{lmttc}{\hyphenchar \font-1 }
\DeclareFontShape{T1}{lmttc}{m}{n}
     {<-> ec-lmtlc10}{}
\DeclareFontShape{T1}{lmttc}{m}{it}
     {<->sub*lmttc/m/sl}{}
\DeclareFontShape{T1}{lmttc}{m}{sl}
     {<-> ec-lmtlco10}{}
\begin{document}
\pagestyle{empty}
\def\myfont{\fontfamily{cmss}\fontseries{lmtt}\selectfont}
\def\mysize{\footnotesize}
\def\mysize{\normalsize}

\def\Y{\textbf{Y}}
\def\r#1{\rotatebox{90}{#1}}
\iffalse
\begin{table}[tb!]
\fontfamily{cmss}\fontseries{lmtt}\selectfont
\caption{Comparison of optimized matrix-vector capabilities}
\begin{tabular}{lccccccc}
\hline
Property & \r{Fortran 9x} & \r{Blitz} & \r{Boost/uBLAS} & \r{Eigen}
& \r{Blaze} & \r{Adept} & \r{valarray (GNU)} \\
\hline
Maximum array rank 
& 7/15 & $n$ & 2 & 2 & 2 & 7 & 1 \\
Bounds checking option 
& \Y & ? & \Y & \Y & ? & \Y & N \\
Fixed-length class 
& \Y & \Y & N? & \Y & \Y & N & N \\
Parallelization 
& \Y? & N & ? & vectorization & ? & N & ? \\
Reference counting 
& N & N? & ? & ? & ? & \Y & N \\
Linear algebra 
& N & N & \Y & \Y & N & \Y & N \\
%Stack class: stack(x)
Linking 
& \Y & N? & ? & ? & ? & \Y & ? \\
Arbitrary templated types 
& N & \Y & \Y & \Y & \Y? & \Y? & \Y\\
Arbitrary storage
& N & \Y & \Y & \Y & N? & \Y & N\\
Sparse
& N & N & \Y & \Y & N & N & N\\
Triangular/banded/symmetric
& N & N & \Y & diag & ? & \Y & N\\
Alias checking
& \Y? & \Y? & manual & \Y & ? & \Y & ?\\
Automatic alias fix
& N & ? & ? N & ? & \Y & ? \\
Arbitrary starting index
& \Y & ? & ? & ? & ? & N & N\\
Algorithmic differentiation
& N & N & N & N & N & \Y & N\\
\hline\end{tabular}
\end{table}
\newpage
\fi
\def\code#1{\texttt{#1}}
\begin{table}[tb!]
%\caption{
\begin{center}
%\parbox{0.9\columnwidth}{
\mysize\myfont Comparison of array syntax between
  Fortran 90 (and later), Matlab and the C++ libraries Adept and Eigen
%In these examples, \code{v} and \code{w} are vectors
%  and \code{A} and \code{B} are matrices.
%}

  \footnotesize
  \myfont
\begin{tabular}{lllll}
\hline
{\large\phantom{X}}
& \mysize Fortran 90+ & \mysize Matlab & \mysize C++ Adept (with C++11 features) & \mysize C++ Eigen \\
\hline
Maximum dimensions &
7 (15 from Fortran 2008) &
Unlimited &
7 &
2
\\
\hline
Vector declaration &
\code{real,dimension(:)} &
&
\code{Vector} &
\code{VectorXd}
\\
Matrix declaration &
\code{real,dimension(:,:)} &
&
\code{Matrix} &
\code{MatrixXd, ArrayXd}
\\
3D array declaration &
\code{real,dimension(:,:,:)}&
&
\code{Array3D}
\\
Fixed matrix declaration &
\code{real,dimension(M,N)} &
&
\code{FixedMatrix<double,false,M,N>} &
\code{Matrix<double,M,N>}
\\
Diagonal matrix declaration&
&
&
\code{DiagMatrix} &
\code{DiagonalMatrix<double,Dynamic>}
\\
%Tridiagonal matrix &
%&
%&
%\code{TridiagMatrix} &
%\\
Symmetric matrix decl.&
&
&
\code{SymmMatrix}
\\
%Upper-triangular matrix &
%&
%&
%\code{UpperMatrix} &
%\\
Sparse matrix declaration&
&
%\code{sparse(A)}
&
&
\code{SparseMatrix<double>}
\\
\hline
Get rank &
\code{rank(A)} &
\code{ndims(A)} &
\code{A::rank}
\\
Get total size &
\code{size(A)} &
\code{numel(A)} &
\code{A.size()} &
\code{A.size()}
\\
Get size of dimension &
\code{size(A,i)} &
\code{size(A,i)} &
\code{A.size(i)} &
\code{A.rows()}, \code{A.cols()}
\\
Get all dimensions &
\code{shape(A)} &
\code{size(A)} &
\code{A.dimensions()}
\\
\hline
Resize &
\code{allocate(A(m,n))} &
\code{A = zeros(m,n)} &
\code{A.resize(m,n)} &
\code{A.resize(m,n)} 
\\
Clear &
\code{deallocate(A)} &
\code{A = []} &
\code{A.clear()} &
\code{A.resize(0,0)}
\\
Link/associate &
\code{A => B} &
&
\code{A >>= B} &
%Low-level access via \code{Map}
(Complicated)
\\
\hline
Set elements to constant &
\code{A = x} &
\code{A(:) = x} &
\code{A = x} &
\code{A.fill(x)}
\\
Fill vector with data &
\code{v = [0,1]} &
\code{v = [0,1]} &
\code{v <{}< 0,1} &
\code{v <{}< 0,1}
\\
Fill matrix with data &
\code{A=reshape([0,1,2,3],[2,2])} &
\code{A = [1 2; 3 4]} &
\code{A <{}< 1,2,3,4} or \code{A = \{\{1,2\},\{3,4\}\}} &
\code{A <{}< 1,2,3,4}
\\
Vector literal &
\code{[1.0, 2.0]} &
\code{[1.0 2.0]} &
\code{Vector\{1.0, 2.0\}} &
\\
\hline
Vector subset &
\code{v(i1:i2)} &
\code{v(i1:i2)} &
\code{v.subset(i1,i2)} &
\code{v.segment(i1,m)}
%\code{Map<VectorXd> w(v.data()+1,8)}
\\
Strided indexing &
\code{v(i1:i2:s)} &
\code{v(i1:s:i2)} &
\code{v(stride(i1,i2,s))} &
%\code{Map<VectorXd,0,InnerStride<> > w(v.data()+1,4,InnerStride<2>)}
(Complicated)
\\
Vector end indexing &
\code{v(i:)} &
\code{v(i:end)} &
\code{v.subset(i,end)} &
\code{v.tail(n)}
\\
Index relative to end &
&
\code{v(end-1)} &
\code{v(end-1)} &
\\
Index by int vector &
\code{v(index)} &
\code{v(index)} &
\code{v(index)}
\\
\hline
Matrix subset &
\code{A(i1:i2,j1:j2)} &
\code{A(i1:i2,j1:j2)} &
\code{A.subset(i1,i2,j1,j2)} &
\code{A.block(i1,j1,m,n)}
\\
Extract row &
\code{A(i,:)} &
\code{A(i,:)} &
\code{A(i,\_\_)} &
\code{A.row(i)}
\\
Matrix end block &
\code{M(i:,j:)} &
\code{M(i:end,j:end)} &
\code{M.subset(i,end,j,end)} &
\code{M.bottomRightCorner(m,n)}
\\
Diagonal matrix from vector &
&
\code{diag(v)} &
\code{v.diag\_matrix()} &
\code{v.asDiagonal()}
\\
Matrix diagonals as vector &
&
\code{diag(A)} &
\code{A.diag\_vector()} &
\code{A.diagonal()} 
\\
Matrix off-diagonals &
&
\code{diag(A,i)} &
\code{A.diag\_vector(i)} &
\code{A.diagonal(i)} 
%\\
%Symmetric view &
%&
%&
%\code{%\color{red}
%A.symm\_matrix<UPPER>()
%}&
%\code{A.selfAdjointView<Upper>()}
%\\
%Upper-triangular view &
%&
%&
%\code{\color{red}A.upper\_matrix()} &
%\code{A.triangularView<Upper>()}
\\
\hline
Elementwise multiplication &
\code{A * B} & 
\code{A .* B} &
\code{A * B} &
\code{A.array() * B.array()}
\\
Elemental function &
\code{sqrt(A)} &
\code{sqrt(A)} &
\code{sqrt(A)} &
\code{A.array().sqrt()}
\\
Addition assignment &
\code{A = A + B} &
\code{A = A + B} &
\code{A += B} &
\code{A.array() += B}
\\
Power &
\code{A ** B} &
\code{A .\textasciicircum\ C} &
\code{pow(A,B)} &
\code{A.array().pow(B)}
\\
\hline
Matrix multiplication &
\code{matmul(A,B)} &
\code{A * B} &
\code{A ** B} &
\code{A * B}
\\
Dot product &
\code{dot\_product(v,w)} &
\code{dot(v,w)} &
\code{dot\_product(v,w)} &
\code{v.dot(w)}
\\
Matrix transpose &
\code{transpose(A)} &
\code{A'} &
\code{A.T()} &
\code{A.transpose()}
\\
In-place transpose &
&
&
\code{A.in\_place\_transpose()} &
\code{A.transposeInPlace()}
\\
Matrix solve &
&
\code{A \textbackslash\ b} &
\code{solve(A,b)} &
\code{A.colPivHouseholderQr().solve(b)}
\\
Matrix inverse &
&
\code{inv(A)} &
\code{inv(A)} &
\code{A.inverse()}
\\
\hline
``Find'' conditional assign &
&
\code{v(find(w<0)) = 0} &
\code{v(find(w<0)) = 0}
\\
``Where'' conditional assign &
\code{where(w<0) v = 0} &
&
\code{v.where(w<0) = 0} &
\code{v = (w<0).select(0,v)}
\\
``Where'' with both cases &
\code{...elsewhere v = 1} &
&
\code{v.where(w<0)=either\_or(0,1)} &
\code{v = (w<0).select(0,1)}
\\
\hline
Average all elements &
\code{mean(A)} & 
\code{mean(A(:)} &
\code{mean(A)} &
\code{A.mean()}
\\
Average along dimension &
\code{mean(A,i)} & 
\code{mean(A,i)} &
\code{mean(A,i)} &
\code{A.colwise().mean()}
\\
Maximum of all elements &
\code{maxval(A)} &
\code{max(A(:))} &
\code{maxval(A)} &
\code{A.maxCoeff()}
\\
Maximum of two arrays &
\code{max(A,B)} &
(Complicated) &
\code{max(A,B)} &
\code{A.max(B)}
\\
Spread along new dimension &
\code{spread(A,dim,n)} &
&
\code{spread<dim>(A,n)}
\\
\hline
\end{tabular}
\end{center}
\end{table}
\end{document}

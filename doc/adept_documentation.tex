% 
% Adept automatic differentiation library for C++: User guide
%
% Type "pdflatex adept_documentation.tex" twice to recreate the PDF
% file (or type "make pdf" in this directory after running the
% configure script one directory above).
%
% Permission is granted to copy, distribute and/or modify this
% document under the terms of the GNU Free Documentation License,
% Version 1.3 or any later version published by the Free Software
% Foundation. This license may be found at
% http://www.gnu.org/copyleft/fdl.html, and in this directory in the
% "COPYING" file. As an exception, no copyright is asserted for the
% code fragments in this document (indicated in the text with a
% light-grey background); these code fragments are in the Public
% Domain and may be copied, modified and distributed without
% restriction.

\documentclass[a4,oneside]{book}
\usepackage[colorlinks=true,linkcolor=blue,citecolor=blue]{hyperref}
\usepackage{natbib}
\usepackage{times}
\usepackage{listings}
\usepackage{xcolor}
\usepackage{color}
\usepackage{marginnote}
\usepackage{rotating}

\usepackage{mdframed,lipsum}
\newmdenv[
  leftmargin = 0pt,
  innerleftmargin = 1em,
  innertopmargin = 0pt,
  innerbottommargin = 0pt,
  innerrightmargin = 0pt,
  rightmargin = 0pt,
  linewidth = 1pt,
  topline = false,
  rightline = false,
  bottomline = false
  ]{leftbar}

% Set math in Times Roman
\DeclareSymbolFont{letters}{OML}{ptmcm}{m}{it}
\DeclareSymbolFont{operators}{OT1}{ptmcm}{m}{n}

% Page set up
\setlength{\oddsidemargin}{0cm} %{0.5cm}
\setlength{\evensidemargin}{0cm} %{0.5cm}
\setlength{\topmargin}{-2cm}
\setlength{\textheight}{24cm}
\setlength{\textwidth}{16cm}
\setlength{\marginparsep}{0.5cm}
\setlength{\marginparwidth}{0cm}
\setlength{\parindent}{1em}
\setlength{\parskip}{0cm}
\renewcommand{\baselinestretch}{1.1}
\sloppy

% Configure appearance of code listings
\definecolor{light-gray}{gray}{0.92}
\def\codesize{\small}
\lstset{language=C++,
  backgroundcolor=\color{light-gray},
  numbersep=5pt,
  xleftmargin=0cm,
  xrightmargin=0cm,
  basicstyle=\footnotesize\ttfamily,
  emph={adouble,xdouble,Stack,adept,Array,FixedArray,Vector,aVector,Matrix,aMatrix,Array3D,aArray3D,intVector,boolVector,floatVector,adept,SpecialMatrix,SquareMatrix,aSquareMatrix,SymmMatrix,aSymmMatrix,UpperMatrix,LowerMatrix,IndexVector},
  emphstyle=\bfseries\color{red}}
\lstset{showstringspaces=false}

% Table-of-contents configuration
\usepackage{tocloft}
\setlength\cftparskip{-2pt}
\setlength\cftbeforesecskip{1pt}
\setlength\cftaftertoctitleskip{2pt}
\renewcommand\cftsecfont{\normalfont}
\renewcommand\cftsecpagefont{\normalfont}
\renewcommand{\cftsecleader}{\cftdotfill{\cftsecdotsep}}
\renewcommand\cftsecdotsep{\cftdot}
\renewcommand\cftsubsecdotsep{\cftdot}

% Page headers
\usepackage{fancyhdr}
\pagestyle{fancy}
\renewcommand{\headrulewidth}{0.5pt}
\renewcommand{\sectionmark}[1]{\markright{\thesection.\ #1}}
\renewcommand{\subsectionmark}[1]{}
\fancyhead[RO,RE]{\thepage}
\fancyfoot[C]{}

% Symbols and macros
\def\x{\ensuremath{{\bf x}}}
\def\y{\ensuremath{{\bf y}}}
\def\H{\ensuremath{{\bf H}}}
\def\T{\ensuremath{^\mathrm{T}}}
\def\Adept{\emph{Adept}}
\def\code#1{{\codesize\texttt{#1}}}
\def\codebf#1{{\codesize\texttt{\textbf{#1}}}}
\def\citem#1{\item[{\codesize\texttt{#1}}]}
\def\codestyle#1{\texttt{#1}}
\def\Offset{size\_t}
\renewcommand\thefootnote{\relax}
\def\cxx11{\marginpar{\rotatebox[origin=rb]{90}{\textbf{C++11 only~~~}}}}
\reversemarginpar

% Title material
\title{\Adept\ C++ Software Library: User Guide}

\author{Robin J. Hogan\\ \emph{European Centre for Medium Range
    Weather Forecasts, Reading, UK}\\ \emph{and School of
    Mathematical, Physical and Computational Sciences, University of
    Reading, UK,}}

\date{Document version 2.0.5 (February 2018) applicable to \Adept\ version
  2.0.5 \thanks{This document is copyright \copyright\ Robin J. Hogan
    2013--2017.  Permission is granted to copy, distribute and/or
    modify this document under the terms of the GNU Free Documentation
    License, Version 1.3 or any later version published by the Free
    Software Foundation. This license may be found at
    \url{http://www.gnu.org/copyleft/fdl.html}.  As an exception, no
    copyright is asserted for the code fragments in this document
    (indicated in the text with a light-grey background); these code
    fragments are hereby placed in the Public Domain, and accordingly
    may be copied, modified and distributed without restriction.}
  \thanks{If you have any queries about \Adept\ that are not answered
    by this document or by the information on the \Adept\ web site
    (\url{http://www.met.reading.ac.uk/clouds/adept/}) then please
    email me at
    \href{mailto:r.j.hogan@ecmwf.int}{\texttt{r.j.hogan@ecmwf.int}}.}}
\begin{document}
\maketitle

\tableofcontents
\def\thefootnote{\fnsymbol{footnote}}
\chapter{Introduction}
\section{What is Adept?}
\Adept\ (Automatic Differentiation using Expression Templates) is a
C++ software library that enables algorithms to be automatically
differentiated. Since version 2.0\footnote{Note that the version 1.9.x
  series served as beta releases for version 2.0 of \Adept.} it also
provides array classes that can be used in array expressions.  These
two capabilities are fully integrated such that array expressions can
be differentiated efficiently, but the array capability may also be
used on its own.

The automatic-differentiation capability uses an operator overloading
approach, so very little code modification is
required. Differentiation can be performed in forward mode (the
``tangent-linear'' computation), reverse mode (the ``adjoint''
computation), or the full Jacobian matrix can be computed. This
behaviour is common to several other libraries, namely ADOL-C
\citep{Griewank+1996}, CppAD \citep{Bell2007} and Sacado
\citep{Gay2005}, but the use of expression templates, an efficient way
to store the differential information and several other optimizations
mean that reverse-mode differentiation tends to be significantly
faster and use less memory. In fact, \Adept\ is also usually only a
little slower than an adjoint code you might write by hand, but
immeasurably faster in terms of user time; adjoint coding is very time
consuming and error-prone. For technical details of how it works,
benchmark results and further discussion of the factors affecting its
speed when applied to a particular code, see \cite{Hogan2014}.

Expression templates also underpin a number of libraries that provide
the capability to perform mathematical operations on entire arrays
\citep{Veldhuizen1995}. Unfortunately, if \Adept\ version 1.x and such
an array library are used together, then the speed advantages of
expression templates are lost, if indeed the libraries work together
at all. Since version 2.0, \Adept\ provides array classes that
overcome this problem: its automatic differentiation and array
capabilities are underpinned by a single unified expression template
framework so that array expressions may be differentiated very
efficiently.  However, it should be stressed that \Adept\ is useful as
a fully functional array library even if you don't wish to use its
automatic differentiation capability. \Adept\ uses BLAS and LAPACK for
matrix operations.

This user guide describes how to apply the \Adept\ software library to
your code, and many of the examples map on to those in the \code{test}
directory of the \Adept\ software package.  Section
\ref{sec:installing} outlines how to install \Adept\ on your system
and how to compile your own code to use it. Chapter \ref{chap:ad}
describes how to use the automatic differentiation capability of the
library, while chapter \ref{chap:arrays} describes how to use its
array capability. Chapter \ref{chap:gen} then describes general
aspects such as exception handling, configuration options and license
terms.

\section{Installing \Adept\ and compiling your code to use it}
\label{sec:installing}
\Adept\ should work with any C++98 compliant compiler, but uses some
C++11 features if compiled with support for this later standard. Most
of the testing has been on Linux with the GNU C++ compiler, but it
also compiles on Linux with the Clang and Intel compilers and on
Windows with the Microsoft compiler. The code is built with the help
of a \code{configure} shell script generated by GNU autotools.  If you
are on a non-Unix system (e.g.\ Windows) and cannot use shell scripts,
see section \ref{sec:non-unix}.
\subsection{Unix-like platforms}
\label{sec:unix}
On a Unix-like system, do the following:
\begin{enumerate}
\item Install the BLAS library to enable matrix multiplication.  For
  the best performance in matrix operations it is recommended that you
  install an optimized package such as OpenBLAS\footnote{OpenBLAS is
    available from \url{http://www.openblas.net/}.} or
  ATLAS\footnote{ATLAS is available from
    \url{http://math-atlas.sourceforge.net/}.}.  If you have multiple
  BLAS libraries available on your system you can specify the one you
  want by calling the \code{configure} script below with
  \code{--with-blas=openblas} or similar.  If \Adept\ is compiled
  without BLAS support then matrix multiplication will fail at run
  time.
\item Optionally install the LAPACK library, necessary for matrix
  inversion and solving linear systems of equations. If you do not
  install this then \Adept\ will still compile but the functions
  \code{inv} and \code{solve} will fail at run time. Note that LAPACK
  relies on the underlying BLAS library for its speed.
\item The test and benchmarking programs can make use of additional
  libraries if available. If you also install any of the automatic
  differentiation tools ADOL-C, CppAD and/or Sacado then the
  benchmarking test program can compare them to \Adept. One of the
  test programs uses the minimization algorithm from the GNU
  Scientific Library, if available, so you may wish to install that
  too.
\item Unpack the package (\code{tar xvfz adept-2.x.tar.gz} on Linux)
  and \code{cd} to the directory \code{adept-2.x}.
\item Configure the build using the \code{configure} script. The most
  basic method is to just run
\begin{lstlisting}
 ./configure
\end{lstlisting}
More likely you will wish to compile with a higher level of
optimization than the default (which is \code{-O2}), achieved by
setting the environment variable \code{CXXFLAGS}. You may also wish to
specify the root directory of the installation, say to
\code{/foo}. These may be done by running instead
\begin{lstlisting}
 ./configure CXXFLAGS="-g -O3" --prefix=/foo
\end{lstlisting}
The \code{-g} option to \code{CXXFLAGS} ensures debugging information
is stored. If you use the GNU compiler then consider the \code{-g1}
option instead to reduce the amount of debugging information
stored. The GNU \code{-march=native} option will also enable the
fastest instruction set for the machine on which the code is being
compiled.  On Intel you may see a performance improvement if AVX
instructions are enabled \code{-mavx} since \Adept\ is able to
vectorize certain expressions. If a library you wish to use is
installed in a non-system directory, say under \code{/foo}, then
specify the locations as follows:
\begin{lstlisting}
 ./configure CPPFLAGS="-I/foo/include" LDFLAGS="-L/foo/lib -Wl,-rpath,/foo/lib"
\end{lstlisting}
where the \code{-rpath} business is needed in order that the
\Adept\ shared library knows where to look for the libraries it is
dependent on.  If you have them then for the benchmarking program you
can also add the non-system location of ADOL-C, CppAD and Sacado
libraries with additional \code{-I} and \code{-L} arguments, but note
that the \code{-rpath} argument is not needed in that case.  You can
see the more general options available by running \code{./configure
  --help}; for example, you can turn-off OpenMP parallelization in the
computation of Jacobian matrices using \code{--disable-openmp}.  See
also section \ref{sec:configuring} for ways to make more fundamental
changes to the configuration of \Adept.  The output from the
\code{configure} script provides information on aspects of how
\Adept\ and the test programs will be built.
\item Build \Adept\ by running
\begin{lstlisting}
 make
\end{lstlisting}
This will create the static and shared libraries in \code{adept/.libs}.
\item Install the header files and the static and shared libraries by
  running
\begin{lstlisting}
 make install
\end{lstlisting}
If this is to be installed to a system directory, you will need to log
in as the super-user first, or run \code{sudo make install} on
depending on your system.
\item Build and run the test programs by running
\begin{lstlisting}
 make check
\end{lstlisting}
Note that this may be done without first installing the
\Adept\ library to a system directory.  This compiles the following
programs in the \code{test} directory:
\begin{itemize}
\item\code{test\_misc}: the trivial example from \cite{Hogan2014};
\item\code{test\_adept}: compares the results of numerical and
  automatic differentiation;
\item\code{test\_with\_without\_ad}: does the same but compiling the
  same source code both with and without automatic differentiation
  (see \code{test/Makefile} for how this is done),
\item\code{test\_radiances}: demonstrates the interfacing of
  \Adept\ with code that provides its own Jacobian;
\item\code{test\_gsl\_interface}: implementation of a simple minimization
  problem using the L-BFGS minimizer in the GSL library;
\item\code{test\_checkpoint}: demonstration of checkpointing, a useful
  technique for large codes;
\item\code{test\_thread\_safe}: demonstration of the use of multiple
  OpenMP threads, each with its own instance of an \Adept\ stack;
\item\code{test\_no\_lib}: demonstrates the use of the
  \code{adept\_source.h} header file that means there is no need to
  link to the \Adept\ library in order to create an executable.
\item\code{test\_arrays}: tests many of the array capabilities
  described in chapter \ref{chap:arrays}.
\item\code{test\_arrays\_active}: tests the same array capabilities
  but with active arrays.
\item\code{test\_arrays\_active\_pasuable}: tests that active array
  operations can be paused as described in setion \ref{sec:pausable}.
\item\code{test\_fixed\_arrays}: tests the similar capabilities of
  arrays with fixed dimensions (known at compile time).
\item\code{test\_fixed\_arrays\_active}: as \code{test\_fixed\_arrays}
  but with active arrays.
\item\code{test\_constructors}: test the different ways of
  constructing, assigning and linking arrays, and passing them to and
  from functions.
\item\code{test\_derivatives}: tests that all mathematical functions
  supported by \Adept\ differentiate correctly.
\item\code{test\_array\_derivatives}: tests that selected array
  operations differentiate correctly.
\item\code{test\_thread\_safe\_arrays}: tests two ways to ensure
  arrays may be accessed and subsetted safely in a multi-threaded
  environment.
\end{itemize}
After compiling these programs, they are run one by one; if any fail
due to an incorrect result then \code{make check} will fail.

The \code{make check} operation also compiles
\code{autodiff\_benchmark} in the \code{benchmark} directory for
comparing the speed of the differentiation of two advection algorithms
using \Adept, ADOL-C, CppAD and Sacado (or whichever subset of these
tools you have on your system).  It also compiles \code{animate} for
visualizing at a terminal what the algorithms are doing.  Further
information on running these programs can be found in the
\code{README} files in the relevant directories.
\end{enumerate}

To compile source files that use the \Adept\ library, you need to make
sure that \code{adept.h} and \code{adept\_arrays.h} are in your
include path. If they are located in a directory that is not in the
default include path, add something like \code{-I/home/fred/include}
to the compiler command line. At the linking stage, add \code{-ladept}
to the command line to tell the linker to look for the
\code{libadept.a} static library, or equivalent shared library. If
this file is in a non-standard location, also add something like
\code{-L/home/fred/lib -Wl,-rpath,/home/fred/lib} before the
\code{-ladept} argument to specify its location. Section
\ref{sec:multipleobjects} provdes an example Makefile for compiling
code that uses the \Adept\ library. Read on to see how you can compile
an \Adept\ application \emph{without} needing to link to a library.

\subsection{Non-Unix platforms, and compiling \Adept\ applications
  without linking to an external library}
\label{sec:non-unix}

Most of the difficulty in maintaining software that can compile on
multiple platforms arises from the different ways of compiling
software libraries, and the need to test on compilers that may be
proprietary.  Unfortunately I don't have the time to maintain versions
of \Adept\ that build specifically on Microsoft Windows or other
non-Unix platforms.  However, \Adept\ is not a large library, so I
have provided a very simple way to build an \Adept\ application
\emph{without} the need to link to a pre-compiled \Adept\ library. In
one of your source files and one only, add this near the top:
\begin{lstlisting}
 #include <adept_source.h>
\end{lstlisting}
Typically you would include this in the source file containing the
\code{main} function.  This header file is simply a concatenation of
the \Adept\ library source files, so when you compile a file that
includes it, you compile in all the functionally of the
\Adept\ library. All other source files in your application should
include only the \code{adept.h} or \code{adept\_arrays.h} header file
as normal.  When you link all your object files together to make an
executable, the \Adept\ functionality will be built in, even though
you did not link to an external \Adept\ library.

By default, \code{adept\_arrays.h} does not enable BLAS (needed for
matrix multiplication) or LAPACK (needed for matrix inversion and
solving linear systems of equations); to enable either BLAS alone, or
both BLAS and LAPACK, uncomment the lines near the top of
\code{adept\_source.h} defining \code{HAVE\_BLAS} and
\code{HAVE\_LAPACK}, and link against functioning BLAS and LAPACK
library. A demonstration of the use of \code{adept\_source.h} is in
the \code{test/test\_no\_lib.cpp} source file, which needs to be
compiled together with \code{test/algorithm.cpp} to make an
executable.
%
It is hoped that this feature will make it easy to use \Adept\ on
non-Unix platforms, although of course this feature works just as well
on Unix-like platforms as well.
%  If you want to use OpenBLAS on such
%platforms then you will still need to install that library in the
%normal way.%

A further point to note is that, under the terms of the license, it is
permitted to copy all the \Adept\ include files, including
\code{adept\_source.h}, into an include directory in your software
package and use it from there in both binary and source-code releases
of your software. This means that users do not need to install
\Adept\ separately before they use your software.  However, if you do
this then remember that your use of these files must comply with the
terms of the Apache License, Version 2.0; see section
\ref{sec:license} for details.
%
\chapter{Using \Adept\ for automatic differentiation}
\label{chap:ad}
%
\section{Introduction}
\label{sec:ad_functionality}
This chapter describes how to use \Adept\ to differentiate your code.
For simplicity, none of the examples use array functionality described
in the next chapter. \Adept\ provides the following
automatic-differentiation functionality:
%
\begin{description}
\item[Full Jacobian matrix] Given the non-linear function $\y=f(\x)$
  relating vector $\y$ to vector $\x$ coded in C or C++, after a
  little code modification \Adept\ can compute the Jacobian matrix
  $\H=\partial\y/\partial\x$, where the element at row $i$ and column $j$ of
  $\H$ is $H_{i,j}=\partial y_i/\partial x_j$. This matrix will be
  computed much more rapidly and accurately than if you simply
  recompute the function multiple times, each time perturbing a
  different element of $\x$ by a small amount. The Jacobian matrix is
  used in the Gauss-Newton and Levenberg-Marquardt minimization
  algorithms.
\item[Reverse-mode differentiation] This is a key component in
  optimization problems where a non-linear function needs to be
  minimized but the state vector $\x$ is too large for it to make
  sense to compute the full Jacobian matrix. Atmospheric data
  assimilation is the canonical example in the field of
  meteorology. Given a nonlinear function $J(\x)$ relating the
  scalar to be minimized $J$ to vector $\x$, \Adept\ will compute the
  vector of adjoints $\partial J/\partial\x$. Moreover, for a
  component of the code that may be expressed as a multi-dimensional
  non-linear function $\y=f(\x)$, \Adept\ can compute $\partial
  J/\partial\x$ if it is provided with the vector of input adjoints
  $\partial J/\partial\y$.  In this case, $\partial J/\partial\x$ is
  equal to the matrix-vector product $\H\T\partial J/\partial\y$, but
  it is computed here without computing the full Jacobian matrix
  $\H$. The vector $\partial J/\partial\x$ may then be used in a
  quasi-Newton minimization scheme \cite[e.g.,][]{Liu+1989}.
\item[Forward-mode differentiation] Given the non-linear function
  $\y=f(\x)$ and a vector of perturbations $\delta\x$, \Adept\ will
  compute the corresponding vector $\delta\y$ arising from a
  linearization of the function $f$. Formally, $\delta\y$ is equal
  to the matrix-vector product $\H\delta\x$, but it is computed here
  without computing the full Jacobian matrix $\H$. Note that
  \Adept\ is designed for the reverse case, so might not be as fast
  or economical in memory in the forward mode as libraries written
  especially for that purpose (although Hogan, 2014, showed that it
  was competitive).
\end{description}%
%
\Adept\ can automatically differentiate the following
operators and functions:
\begin{itemize}
\item The standard binary mathematical operators \code{+}, \code{-},
  \code{*} and \code{/}.
\item The assignment versions of these operators:
  \code{+=}, \code{-=}, \code{*=} and \code{/=}.
\item The unary mathematical functions \code{sqrt}, \code{exp},
  \code{log}, \code{log10}, \code{sin}, \code{cos}, \code{tan},
  \code{asin}, \code{acos}, \code{atan}, \code{sinh}, \code{cosh},
  \code{tanh}, \code{abs}, \code{asinh}, \code{acosh}, \code{atanh},
  \code{expm1}, \code{log1p}, \code{cbrt}, \code{erf}, \code{erfc},
  \code{exp2}, \code{log2}, \code{round}, \code{trunc}, \code{rint}
  and \code{nearbyint},
\item The binary functions \code{pow}, \code{atan2}, \code{min},
  \code{max}, \code{fmin} and \code{fmax}.
\end{itemize}
Variables to take part in expressions to be differentiated have a
special ``active'' type; such variables can take part in comparison
operations \code{==}, \code{!=}, \code{>}, \code{<}, \code{>=} and
\code{<=}, as well as the diagnostic functions \code{isfinite},
\code{isinf} and \code{isnan}.

Note that at present \Adept\ is missing some functionality that you may
require:

\begin{itemize}
\item Differentiation is first-order only: it cannot directly compute
  higher-order derivatives such as the Hessian matrix (although one of
  the Frequently Asked Questions in section \ref{sec:faq} describes
  how \Adept\ can help compute the Hessian of a certain category of
  algorithms).
\item It has limited support for complex numbers; no support for
  mathematical functions of complex numbers, and expressions involving
  operations (addition, subtraction, multiplication and division) on
  complex numbers are not optimized.
\item It can be applied to C and C++ only; \Adept\ could not be
  written in Fortran since the language provides no template
  capability.
\end{itemize}%
%
It is hoped that future versions will remedy these limitations (and
maybe even a future version of Fortran will support templates).

Section \ref{sec:preparation} describes how to prepare your code for
automatic differentiation, and section \ref{sec:adjoint} describes how
to perform forward- and reverse-mode automatic differentiation on this
code. Section \ref{sec:jacobian} describes how to compute Jacobian
matrices. Section \ref{sec:realworld} provides a detailed description
of how to interface an algorithm implemented using \Adept\ with a
third-party minimization library.  Section \ref{sec:withwithout}
describes how to call a function both with and without automatic
differentiation from within the same program. Section
\ref{sec:interfacehandcoded} describes how to interface to software
modules that compute their own Jacobians.  Section \ref{sec:stack}
describes the user-oriented member functions of the \code{Stack} class
that contains the differential information and section
\ref{sec:adouble} describes the member functions of the ``active''
double-precision type \code{adouble}.


\section{Code preparation}
\label{sec:preparation}
If you have used ADOL-C, CppAD or Sacado then you will already be
familiar with what is involved in applying an operator-overloading
automatic differentiation package to your code. The user interface to
\Adept\ differs from these only in the detail. It is assumed that you
have an algorithm written in C or C++ that you wish to
differentiate. This section deals with the modifications needed to
your code, while section \ref{sec:adjoint} describes the small
additional amount of code you need to write to differentiate it.

In all source files containing code to be differentiated, you need to
include the \code{adept.h} header file and import the \code{adouble}
type from the \code{adept} namespace. Assuming your code uses double
precision, you then search and replace \code{double} with the
``active'' equivalent \code{adouble}, but doing this only for those
variables whose values depend on the independent input variables.
Under the hood this type is an alias for \code{Active<double>}.  The
single-precision equivalent is \code{afloat}, an alias for
\code{Active<float>}.  Active and passive variables of single and
double precision may be used together in the same expressions, but
note that by default all differentiation is done in double precision.

If you wish to enable your code to be easily recompiled to use
different precisions, then you may alternatively use the generic
\code{Real} type from the \code{adept} namespace with its active
equivalent \code{aReal} (an alias for \code{Active<Real>}). Section
\ref{sec:configuring} describes how to redefine \code{Real} to
represent single, double or quadruple precision.  Automatic
differentiation will be performed using the same precision as
\code{Real}, but but be aware that if this is defined to be the same
as a single-precision \code{float}, accumulation of round-off error
can make the accuracy of derivatives insufficient for minimization
algorithms. The examples in the remainder of this chapter use only
double precision.

Consider the following contrived algorithm from \cite{Hogan2014} that
takes two inputs and returns one output:

\begin{lstlisting}
 double algorithm(const double x[2]) {
   double y = 4.0;
   double s = 2.0*x[0] + 3.0*x[1]*x[1];
   y *= sin(s);
   return y;
 }
\end{lstlisting}

\noindent The modified code would look like this:

\begin{lstlisting}
 #include <adept.h>
 using adept::adouble;

 adouble algorithm(const adouble x[2]) {
   adouble y = 4.0;
   adouble s = 2.0*x[0] + 3.0*x[1]*x[1];
   y *= sin(s);
   return y;
 }
\end{lstlisting}

\noindent Changes like this need to be done in all source files that
form part of an algorithm to be differentiated. 

If you need to access the real number underlying an \code{adouble}
variable \code{a}, for example in order to use it as an argument to
the \code{fprintf} function, then use \code{a.value()} or
\code{adept::value(a)}. Any mathematical operations performed on
this real number will not be differentiated.

You may use \code{adouble} as the template argument of a Standard
Template Library (STL) vector type (i.e.  \code{std::vector\textless
  adouble\textgreater}), or indeed any container where you access
individual elements one by one. For types allowing mathematical
operations on the whole object, such as the STL \code{complex} and
\code{valarray} types, you will find that although you can multiply
one \code{std::complex\textless adouble\textgreater} or
\code{std::valarray\textless adouble\textgreater} object by another,
mathematical functions (\code{exp}, \code{sin} etc.) will not work
when applied to whole objects, and neither will some simple operations
such as multiplying these types by an ordinary (non-active)
\code{double} variable.  Moreover, the performance is not great
because expressions cannot be fully optimized when in these
containers.  Therefore If you need array functionality then you should
use the features described in chapter \ref{chap:arrays}.  It is hoped
that a future version of \Adept\ will include its own complex type.

\section{Applying reverse-mode differentiation}
\label{sec:adjoint}

Suppose you wanted to create a version of \code{algorithm} that
returned not only the result but also the gradient of the result with
respect to its inputs, you would do this:

\begin{lstlisting}
 #include <adept.h>
 double algorithm_and_gradient(
                     const double x_val[2], // Input values
                     double dy_dx[2]) {     // Output gradients
   adept::Stack stack;                      // Where the derivative information is stored
   using adept::adouble;                    // Import adouble from adept
   adouble x[2] = {x_val[0], x_val[1]};     // Initialize active input variables
   stack.new_recording();                   // Start recording
   adouble y = algorithm(x);                // Call version overloaded for adouble args
   y.set_gradient(1.0);                     // Defines y as the objective function 
   stack.compute_adjoint();                 // Run the adjoint algorithm
   dy_dx[0] = x[0].get_gradient();          // Store the first gradient
   dy_dx[1] = x[1].get_gradient();          // Store the second gradient
   return y.value();                        // Return the result of the simple computation
 }
\end{lstlisting}
%
The component parts of this function are in a specific order, and if
this order is violated then the code will not run correctly. The steps
are now described.
%
\subsection{Set-up stack to record derivative information}
\label{sec:stack_setup}
\begin{lstlisting}
 adept::Stack stack;
\end{lstlisting}
The \code{Stack} object is where the differential version of the
algorithm will be stored. When initialized, it makes itself accessible
to subsequent statements via a global variable, but using thread-local
storage to ensure thread safety. \emph{It must be initialized before
  the first \code{adouble} object is instantiated and it must not go
  out of scope until the last \code{adouble} object is destructed.}
This is because \code{adouble} objects register themselves with the
currently active stack, and deregister themselves when they are
destroyed; if the same stack is not active throughout the lifetime of
such \code{adouble} objects then the code will crash with a
segmentation fault.

In the example here, the \code{Stack} object is local to the scope of
the function. If another \code{Stack} object had been initialized by
the calling function and so was active at the point of entry to the
function, then the local \code{Stack} object would throw an
\code{adept::stack\_already\_active} exception (see Test 3 described
at \code{test/README} in the \Adept\ package if you want to use
multiple \code{Stack} objects in the same program).  A disadvantage of
local \code{Stack} objects is that the memory it uses must be
reallocated each time the function is called.  This can be overcome in
several ways:
\begin{itemize}
\item Declare the \code{Stack} object to be \code{static}, which means
  that it will persist between function calls. This has the
  disadvantage that you won't be able to use other \code{Stack}
  objects in the program without deactivating this one first (see Test
  3 in the \Adept\ package, referred to above, for how to do this).
\item Initialize \code{Stack} at a higher level in the program. If you
  need access to the stack, you may either pass a reference to it to
  functions such as \code{algorithm\_and\_gradient}, or alternatively
  you can use the \code{adept::active\_stack()} function to return a
  pointer to the currently active stack object.
\item Put it in a class so that it is accessible to member functions;
  this approach is demonstrated in section \ref{sec:realworld}.
\end{itemize}
%
\subsection{Initialize independent variables and start recording}
\begin{lstlisting}
 adouble x[2] = {x_val[0], x_val[1]};
 stack.new_recording();
\end{lstlisting}
The first line here simply copies the input values to the algorithm
into \code{adouble} variables. These are the \emph{independent
  variables}, but note that there is no obligation for these to be
stored as one array (as in CppAD), and for forward- and reverse-mode
automatic differentiation you do not need to tell \Adept\ explicitly
via a function call which variables are the independent ones. The next
line clears all differential statements from the stack so that it is
ready for a new recording of differential information.
%
Note that the first line here actually stores two differential
statements, $\delta$\code{x[0]=0} and $\delta$\code{x[1]=0}, which are
immediately cleared by the \code{new\_recording} function call.  To
avoid the small overhead of storing redundant information on the
stack, we could replace the first line with 
\begin{lstlisting}
 x[0].set_value(x_val[0]);
 x[1].set_value(x_val[1]);
\end{lstlisting}
or
\begin{lstlisting}
 adept::set_values(x, 2, x_val);
\end{lstlisting}
which have the effect of setting the values of \code{x} without storing
the equivalent differential statements.

Previous users of \Adept\ version 0.9 should note that since version
1.0, the \code{new\_recording} function replaces the \code{start}
function call, which had to be put \emph{before} the independent
variables were initialized.  The problem with this was that the
independent variables had to be initialized with the \code{set\_value}
or \code{set\_values} functions, otherwise the gradients coming out of
the automatic differentiation would all be zero.  Since it was easy to
forget this, \code{new\_recording} was introduced to allow the
independent variables to be assigned in the normal way using the
assignment operator (\code{=}).  But don't just replace \code{start}
in your version-0.9-compatible code with \code{new\_recording}; the
latter must appear \emph{after} the independent variables have been
initialized.

\subsection{Perform calculations to be differentiated}
\begin{lstlisting}
 adouble y = algorithm(x);
\end{lstlisting}
The algorithm is called, and behind the scenes the equivalent
differential statement for every mathematical statement is stored in the
stack. The result of the forward calculation is stored in \code{y},
known as a dependent variable. This example has one dependent
variable, but any number is allowed, and they could be returned in
another way, e.g. by passing a non-constant array to algorithm that is
filled with the final values when the function returns.
%
\subsection{Perform reverse-mode differentiation}

\begin{lstlisting}
 y.set_gradient(1.0);
 stack.compute_adjoint();
\end{lstlisting}
The first line sets the initial gradient (or adjoint) of \code{y}. In
this example, we want the output gradients to be the derivatives of
\code{y} with respect to each of the independent variables; to achieve
this, the initial gradient of \code{y} must be unity.

More generally, if \code{y} was only an intermediate value in the
computation of objective function $J$, then for the outputs of the
function to be the derivatives of $J$ with respect to each of the
independent variables, we would need to set the gradient of
\code{y} to $\partial J/\partial$\code{y}. In the case of multiple
intermediate values, a separate call to \code{set\_gradient} is needed
for each intermediate value.  If \code{y} was an array of length
\code{n} then the gradient of each element could be set to the values in a \code{double} array \code{y\_ad} using
\begin{lstlisting}
 adept::set_gradients(y, n, y_ad);
\end{lstlisting}

The \code{compute\_adjoint()} member function of stack performs the
adjoint calculation, sweeping in reverse through the differential
statements stored on the stack. Note that this must be preceded by at
least one \code{set\_gradient} or \code{set\_gradients} call, since
the first such call initializes the list of gradients for
\code{compute\_adjoint()} to act on. Otherwise,
\code{compute\_adjoint()} will throw a
\code{gradients\_not\_initialized} exception. 

\subsection{Extract the final gradients}

\begin{lstlisting}
 dy_dx[0] = x[0].get_gradient();
 dy_dx[1] = x[1].get_gradient();
\end{lstlisting}
These lines simply extract the gradients of the objective function
with respect to the two independent variables. Alternatively we could
have extracted them simultaneously using
\begin{lstlisting}
 adept::get_gradients(x, 2, dy_dx);
\end{lstlisting}

To do forward-mode differentiation in this example would involve
setting the initial gradients of \code{x} instead of \code{y}, calling
the member function \code{compute\_tangent\_linear()} instead of
\code{compute\_adjoint()}, and extracting the final gradients from
\code{y} instead of \code{x}.

\section{Computing Jacobian matrices}
\label{sec:jacobian}
Until now we have considered a function with two inputs and one
output.  Consider the following more general function whose declaration
is
\begin{lstlisting}
 void algorithm2(int n, const adouble* x, int m, adouble* y);
\end{lstlisting}
where \code{x} points to the \code{n} independent (input) variables
and \code{y} points to the \code{m} dependent (output) variables. The
following function would return the full Jacobian matrix:
%
\begin{lstlisting}
 #include <vector>
 #include <adept.h>
 void algorithm2_jacobian(
                     int n,                 // Number of input values
                     const double* x_val,   // Input values
                     int m,                 // Number of output values
                     double* y_val,         // Output values
                     double* jac) {         // Output Jacobian matrix
   using adept::adouble;                    // Import Stack and adouble from adept
   adept::Stack stack;                      // Where the derivative information is stored
   std::vector<adouble> x(n);               // Vector of active input variables
   adept::set_values(&x[0], n, x_val);      // Initialize adouble inputs
   adept.new_recording();                   // Start recording
   std::vector<adouble> y(m);               // Create vector of active output variables
   algorithm2(n, &x[0], m, &y[0]);          // Run algorithm
   stack.independent(&x[0], n);             // Identify independent variables
   stack.dependent(&y[0], m);               // Identify dependent variables
   stack.jacobian(jac);                     // Compute & store Jacobian in jac
 }
\end{lstlisting}
%
Note that:
\begin{itemize}
\item The \code{independent} member function of stack is used to
  identify the independent variables, i.e.\ the variables that the
  derivatives in the Jacobian matrix will be with respect to. In this
  example there are \code{n} independent variables located together in
  memory and so can be identified all at once. Multiple calls are
  possible to identify further independent variables.  To identify a
  single independent variable, call \code{independent} with just one
  argument, the independent variable (not as a pointer). 
\item The \code{dependent} member function of stack identifies the
  dependent variables, and its usage is identical to
  \code{independent}.
\item The memory provided to store the Jacobian matrix (pointed to by
  \code{jac}) must be a one-dimensional array of size
  \code{m}$\times$\code{n}, where \code{m} is the number of dependent
  variables and \code{n} is the number of independent variables.
\item The resulting matrix is stored in the sense of the index
  representing the dependent variables varying fastest (column-major
  order).
% To get row-major order, call the \code{jacobian} function
%  with a second argument of \code{true} (see section \ref{sec:stack}).
\item Internally, the Jacobian calculation is performed by multiple
  forward or reverse passes, whichever would be faster (dependent on
  the numbers of independent and dependent variables).
\item The use of \code{std::vector<adouble>} rather than \code{new
  adouble[n]} ensures no memory leaks in the case of an exception being
  thrown, since the memory associated with \code{x} and \code{y} will
  be automatically deallocated when they go out of scope.
\end{itemize}%

\section{Real-world usage: interfacing \Adept\ to a minimization library}
\label{sec:realworld}
Suppose we want to find the vector $\x$ that minimizes an objective function
$J(\x)$ that consists of a large algorithm coded using the
\Adept\ library and encapsulated within a C++ class.  In this section
we illustrate how it may be interfaced to a third-party minimization
algorithm with a C-style interface, specifically the free one in the
GNU Scientific Library.  The full working version of this example,
using the N-dimensional Rosenbrock banana function as the function to
be minimized, is ``Test 4'' in the \code{test} directory of the
\Adept\ software package. The interface to the algorithm is as
follows:
%
\begin{lstlisting}
 #include <vector>
 #include <adept.h>
 using adept::adouble;
 class State {
  public:
    // Construct a state with n state variables
    State(int n) { active_x_.resize(n); x_.resize(n); }
    // Minimize the function, returning true if minimization successful, false otherwise
    bool minimize();
    // Get copy of state variables after minimization
    void x(std::vector<double>& x_out) const;
    // For input state variables x, compute the function J(x) and return it
    double calc_function_value(const double* x);
    // For input state variables x, compute function and put its gradient in dJ_dx
    double calc_function_value_and_gradient(const double* x, double* dJ_dx);
    // Return the size of the state vector
    unsigned int nx() const { return active_x_.size(); }
  protected:
    // Active version: the algorithm is contained in the definition of this function
    adouble calc_function_value(const adouble* x);
    // DATA
    adept::Stack stack_;             // Adept stack object
    std::vector<adouble> active_x_;  // Active state variables
 };
\end{lstlisting}
%
The algorithm itself is contained in the definition of
\code{calc\_function\_value(const adouble*)}, which is implemented using
\code{adouble} variables (following the rules in section
\ref{sec:preparation}). However, the public interface to the class
uses only standard \code{double} types, so the use of \Adept\ is
hidden to users of the class.  Of course, a complicated algorithm may
be implemented in terms of multiple classes that do exchange data via
\code{adouble} objects. We will be using a quasi-Newton minimization
algorithm that calls the algorithm many times with trial vectors $\x$,
and for each call may request not only the value of the function, but
also its gradient with respect to $\x$. Thus the public interface
provides \code{calc\_function\_value(const double*)} and
\code{calc\_function\_value\_and\_gradient}, which could be implemented as
follows:
%
\begin{lstlisting}
 double State::calc_function_value(const double* x) {
   for (unsigned int i = 0; i < nx(); ++i) active_x_[i] = x[i];
   stack_.new_recording();
   return value(calc_function_value(&active_x_[0]));
 }

 double State::calc_function_value_and_gradient(const double* x, double* dJ_dx) {
   for (unsigned int i = 0; i < nx(); ++i) active_x_[i] = x[i];
   stack_.new_recording();
   adouble J = calc_function_value(&active_x_[0]);
   J.set_gradient(1.0);
   stack_.compute_adjoint();
   adept::get_gradients(&active_x_[0], nx(), dJ_dx);
   return value(J);
 }
\end{lstlisting}
%
The first function simply copies the \code{double} inputs into an
\code{adouble} vector and runs the version of
\code{calc\_function\_value} for \code{adouble} arguments. Obviously
there is an inefficiency here in that gradients are recorded that are
then not used, and this function would be typically 2.5--3 times
slower than an implementation of the algorithm that did not store
gradients.  Section \ref{sec:withwithout} describes three ways to
overcome this problem.  The second function above implements
reverse-mode automatic differentiation as described in section
\ref{sec:adjoint}.

The \code{minimize} member function could be implemented using GSL as
follows:
%
\begin{lstlisting}
 #include <iostream>
 #include <gsl/gsl_multimin.h>

 bool State::minimize() {
   // Minimizer settings
   const double initial_step_size = 0.01;
   const double line_search_tolerance = 1.0e-4;
   const double converged_gradient_norm = 1.0e-3;
   // Use the "limited-memory BFGS" quasi-Newton minimizer
   const gsl_multimin_fdfminimizer_type* minimizer_type
     = gsl_multimin_fdfminimizer_vector_bfgs2;

   // Declare and populate structure containing function pointers
   gsl_multimin_function_fdf my_function;
   my_function.n = nx();
   my_function.f = my_function_value;
   my_function.df = my_function_gradient;
   my_function.fdf = my_function_value_and_gradient;
   my_function.params = reinterpret_cast<void*>(this);
   
   // Set initial state variables using GSL's vector type
   gsl_vector *x;
   x = gsl_vector_alloc(nx());
   for (unsigned int i = 0; i < nx(); ++i) gsl_vector_set(x, i, 1.0);

   // Configure the minimizer
   gsl_multimin_fdfminimizer* minimizer
     = gsl_multimin_fdfminimizer_alloc(minimizer_type, nx());
   gsl_multimin_fdfminimizer_set(minimizer, &my_function, x,
                                 initial_step_size, line_search_tolerance);
   // Begin loop
   size_t iter = 0;
   int status;
   do {
     ++iter;
     // Perform one iteration
     status = gsl_multimin_fdfminimizer_iterate(minimizer);

     // Quit loop if iteration failed
     if (status != GSL_SUCCESS) break;
    
     // Test for convergence
     status = gsl_multimin_test_gradient(minimizer->gradient, converged_gradient_norm);
   }
   while (status == GSL_CONTINUE && iter < 100);

   // Free memory
   gsl_multimin_fdfminimizer_free(minimizer);
   gsl_vector_free(x);

   // Return true if successfully minimized function, false otherwise
   if (status == GSL_SUCCESS) {
     std::cout << "Minimum found after " << iter << " iterations\n";
     return true;
   }
   else {
     std::cout << "Minimizer failed after " << iter << " iterations: "
               << gsl_strerror(status) << "\n";
     return false;
   }
 }
\end{lstlisting}
%
The GSL interface requires three functions to be defined, each of
which takes a vector of state variables $\x$ as input:
\code{my\_function\_value}, which returns the value of the function;
\code{my\_function\_gradient}, which returns the gradient of the
function with respect to $\x$; and
\code{my\_function\_value\_and\_gradient}, which returns the value and
the gradient of the function. These functions are provided to GSL as
function pointers (see above), but since GSL is a C library, we need
to use the `\code{extern "C"}' specifier in their definition. Thus the
function definitions would be:
%
\begin{lstlisting}
 extern "C" 
 double my_function_value(const gsl_vector* x, void* params) {
   State* state = reinterpret_cast<State*>(params);
   return state->calc_function_value(x->data);
 }

 extern "C"
 void my_function_gradient(const gsl_vector* x, void* params, gsl_vector* gradJ) { 
   State* state = reinterpret_cast<State*>(params);
   state->calc_function_value_and_gradient(x->data, gradJ->data);
 }

 extern "C"
 void my_function_value_and_gradient(const gsl_vector* x, void* params,
                                     double* J, gsl_vector* gradJ) { 
   State* state = reinterpret_cast<State*>(params);
   *J = state->calc_function_value_and_gradient(x->data, gradJ->data);
 }
\end{lstlisting}
%
When the \code{gsl\_multimin\_fdfminimizer\_iterate} function is
called, it chooses a search direction and performs several calls of
these functions to approximately minimize the function along this
search direction. The \code{this} pointer (i.e.\ the pointer to the
\code{State} object), which was provided to the \code{my\_function}
structure in the definition of the \code{minimize} function above, is
provided as the second argument to each of the three functions
above. Unlike in C, in C++ this pointer needs to be cast back to a
pointer to a \code{State} type, hence the use of
\code{reinterpret\_cast}.

That's it! A call to \code{minimize} should successfully minimize well
behaved differentiable multi-dimensional functions.  It should be
straightforward to adapt the above to work with other minimization
libraries.

\section{Calling an algorithm with and without automatic differentiation from the same program}
\label{sec:withwithout}
The \code{calc\_function\_value(const double*)} member function
defined in section \ref{sec:realworld} is sub-optimal in that it
simply calls the \code{calc\_function\_value(const adouble*)} member
function, which not only computes the value of the function, it also
records the derivative information of all the operations involved.
This information is then ignored. This overhead makes the function
typically 2.5--3 times slower than it needs to be, although sometimes
(specifically for loops containing no trancendental functions) the
difference between an algorithm coded in terms of \code{double}s and
the same algorithm coded in terms of \code{adouble}s can exceed a
factor of 10 \citep{Hogan2014}.  The impact on the computational speed
of the entire minimization process depends on how many requests are
made for the function value only as opposed to the gradient of the
function, and can be significant.  We require a way to avoid the
overhead of \Adept\ computing the derivative information for calls to
\code{calc\_function\_value(const double*)}, without having to
maintain two versions of the algorithm, one coded in terms of
\code{double}s and the other in terms of \code{adouble}s. The three
ways to achieve this are now described.
%
\subsection{Function templates}
\label{sec:func_templates}
The simplest approach is to use a function template for those
functions that take active arguments, as demonstrated in the following
example:
%
\begin{lstlisting}
 #include <adept.h>
 class State {
  public:
    ...
    template <typename xdouble>
    xdouble calc_function_value(const xdouble* x);
    ...
 };

 // Example function definition that must be in a header file included
 // by any source file that calls calc_function_value
 template <typename xdouble>
 inline
 xdouble State::calc_function_value(const xdouble* x) {
   xdouble y = 4.0;
   xdouble s = 2.0*x[0] + 3.0*x[1]*x[1];
   y *= sin(s);
   return y;
 }
\end{lstlisting}
%
This takes the example from section \ref{sec:preparation} and replaces
\code{adouble} by the template type \code{xdouble}. Thus,
\code{calc\_function\_value} can be called with either \code{double}
or \code{adouble} arguments, and the compiler will compile inline the
inactive or active version accordingly.  Note that the function
template need not be a member function of a class.  

This technique is good if only a small amount of code needs to be
differentiated, but for large models the use of inlining is likely to
lead to duplication of compiled code leading to large executables and
long compile times.  The following two approaches do not have this
drawback and are suitable for large codes.

\subsection{Pausable recording}
\label{sec:pausable}
The second method involves compiling the entire code with the
\code{ADEPT\_RECORDING\_PAUSABLE} preprocessor variable defined, which
can be done by adding an argument \code{-DADEPT\_RECORDING\_PAUSABLE}
to the compler command line. This modifies the behaviour of
mathematical operations performed on \code{adouble} variables: instead
of performing the operation and then storing the derivative
information, it performs the operation and then only stores the
derivative information if the \Adept\ stack is not in the ``paused''
state. We then use the following member function definition instead of
the one in section \ref{sec:realworld}:
%
\begin{lstlisting}
 double State::calc_function_value(const double* x) {
   stack_.pause_recording();
   for (unsigned int i = 0; i < nx(); ++i) active_x_[i] = x[i];
   double J = value(calc_function_value(&active_x_[0]));
   stack_.continue_recording();
   return J;
 }
\end{lstlisting}
%
By pausing the recording for all operations on \code{adouble} objects,
most of the overhead of storing derivative information is removed. The
extra run-time check to see whether the stack is in the paused state,
which is carried out by mathematical operations involving
\code{adouble} objects, generally adds a small overhead.  However, in
algorithms where most of the number crunching occurs in loops
containing no trancendental functions, even if the stack is in the
paused state, the presence of the check can prevent the compiler from
agressively optimizing the loop.  In that instance the third method
may be preferable.
%
\subsection{Multiple object files per source file}
\label{sec:multipleobjects}
The third method involves compiling each source file containing
functions with \code{adouble} arguments twice.  The first time, the
code is compiled normally to produce an object file containing
compiled functions including automatic differentiation. The second
time, the code is compiled with the
\code{-DADEPT\_NO\_AUTOMATIC\_DIFFERENTIATION} flag on the compiler
command line. This instructs the \code{adept.h} header file to turn
off automatic differentiation by defining the \code{adouble} type to
be an alias of the \code{double} type. This way, a second set of
object files are created containing overloaded versions of the same
functions as the first set but this time without automatic
differentiation. These object files can be compiled together to form
one executable.  In the example presented in section
\ref{sec:realworld}, the \code{calc\_function\_value} function would
be one that would be compiled twice in this way, once to provide the
\code{calc\_function\_value(const adouble*)} version and the other to
provide the \code{calc\_function\_value(const double*)} version. Note
that any functions that do not include \code{adouble} arguments must
be compiled only once, because otherwise the linker will complain
about multiple versions of the same function.

The following shows a Makefile from a hypothetical project that
compiles two source files (\code{algorithm1.cpp} and
\code{algorithm2.cpp}) twice and a third (\code{main.cpp}) once:
%
\begin{lstlisting}[language=make]
 # Specify compiler and flags
 CXX = g++
 CXXFLAGS = -Wall -O3 -g
 # Normal object files to be created
 OBJECTS = algorithm1.o algorithm2.o main.o
 # Object files created with no automatic differentiation
 NO_AD_OBJECTS = algorithm1_noad.o algorithm2_noad.o
 # Program name
 PROGRAM = my_program
 # Include-file location
 INCLUDES = -I/usr/local/include
 # Library location and name, plus the math library
 LIBS = -L/usr/local/lib -lm -ladept

 # Rule to build the program (typing "make" will use this rule)
 $(PROGRAM): $(OBJECTS) $(NO_AD_OBJECTS)
         $(CXX) $(CXXFLAGS) $(OBJECTS) $(NO_AD_OBJECTS) $(LIBS) -o $(PROGRAM)
 # Rule to build a normal object file (used to compile all objects in OBJECTS)
 %.o: %.cpp
         $(CXX) $(CXXFLAGS) $(INCLUDES) -c $<
 # Rule to build a no-automatic-differentiation object (used to compile ones in NO_AD_OBJECTS)
 %_noad.o: %.cpp
         $(CXX) $(CXXFLAGS) $(INCLUDES) -DADEPT_NO_AUTOMATIC_DIFFERENTIATION -c $< -o $@
\end{lstlisting}
%

There is a further modification required with this approach, which
arises because if a header file declares both the \code{double} and
\code{adouble} versions of a function, then when compiled with
\code{-DADEPT\_NO\_AUTOMATIC\_DIFFERENTIATION} it appears to the
compiler that the same function is declared twice, leading to a
compile-time error.  This can be overcome by using the preprocessor to
hide the \code{adouble} version if the code is compiled with this
flag, as follows (using the example from section \ref{sec:realworld}):
%
\begin{lstlisting}
 #include <adept.h>
 class State {
  public:
    ...
    double calc_function_value(const double* x);
  protected:
 #ifndef ADEPT_NO_AUTOMATIC_DIFFERENTIATION
    adouble calc_function_value(const adouble* x);
 #endif
    ...
 };
\end{lstlisting}

A final nuance is that if the code contains an \code{adouble} object
\code{x}, then \code{x.value()} will work fine in the compilation when
\code{x} is indeed of type \code{adouble}, but in the compilation when
it is set to a simple \code{double} variable, the \code{value()}
member function will not be found.  Hence it is better to use
\code{adept::value(x)}, which returns a \code{double} regardless of
the type of \code{x}, and works regardless of whether the code was
compiled with or without the
\code{-DADEPT\_NO\_AUTOMATIC\_DIFFERENTIATION} flag.

\section{Interfacing with software containing hand-coded Jacobians}
\label{sec:interfacehandcoded}
Often a complicated algorithm will include multiple components.
Components of the code written in C or C++ for which the source is
available are straightforward to convert to using \Adept, following
the rules in section \ref{sec:preparation}.  For components written in
Fortran, this is not possible, but if such components have their own
hand-coded Jacobian then it is possible to interface \Adept\ to them.
More generally, in certain situations automatic differentiation is
much slower than hand-coding \cite[see the Lax-Wendroff example
  in][]{Hogan2014} and we may wish to hand-code certain critical
parts.  In general the Jacobian matrix is quite expensive to compute,
so this interfacing strategy makes most sense if the component of the
algorithm has a small number of inputs or a small number of
outputs. A full working version of the following example is given as
``Test 3'' in the \code{test} directory of the \Adept\ package.

Consider the example of a radiative transfer model for simulating
satellite microwave radiances at two wavelengths, $I$ and $J$, which
takes as input the surface temperature $T_s$ and the vertical profile
of atmospheric temperature $T$ from a numerical weather forecast
model. Such a model would be used in a data assimilation system to
assimilate the temperature information from the satellite observations
into the weather forecast model. In addition to returning the
radiances, the model returns the gradient $\partial I/\partial T_s$
and the gradients $\partial I/\partial T_i$ for all height layers $i$
between 1 and $n$, and likewise for radiance $J$. The interface to the
radiative transfer model is the following:
%
\begin{lstlisting}
 void simulate_radiances(int n, // Size of temperature array
                         // Input variables:
                         double surface_temperature, 
                         const double* temperature,
                         // Output variables:
                         double radiance[2],
                         // Output Jacobians:
                         double dradiance_dsurface_temperature[2],
                         double* dradiance_dtemperature);
\end{lstlisting}
%
The calling function needs to allocate \code{2*n} elements for the
temperature Jacobian \code{dradiance\_dtemperature} to be stored, and
the stored Jacobian will be oriented such that the radiance index
varies fastest.

\Adept\ needs to be told how to relate the radiance perturbations
$\delta I$ and $\delta J$, to perturbations in the input
variables, $\delta T_s$ and $\delta T_i$ (for all layers
$i$). Mathematically, we wish the following relationship to be stored
within the \Adept\ stack:
%
\begin{equation}
\delta I = \frac{\partial I}{\partial T_s}\delta
T_s+\sum_{i=1}^n\frac{\partial I}{\partial T_i}\delta T_i.\nonumber
\end{equation}
%
This is achieved with the following wrapper function, which has
\code{adouble} inputs and outputs and therefore can be called from
within other parts of the algorithm that are coded in terms of
\code{adouble} objects:
%
\begin{lstlisting}
 void simulate_radiances_wrapper(int n,
                                 const adouble& surface_temperature,
                                 const adouble* temperature,
                                 adouble radiance[2]) {
   // Create inactive (double) versions of the active (adouble) inputs
   double st = value(surface_temperature);
   std::vector<double> t(n);
   for (int i = 0; i < n; ++i) t[i] = value(temperature[i]);

   // Declare variables to hold the inactive outputs and their Jacobians
   double r[2];
   double dr_dst[2];
   std::vector<double> dr_dt(2*n);

   // Call the non-Adept function
   simulate_radiances(n, st, &t[0], &r[0], dr_dst, &dr_dt[0]);

   // Copy the results into the active variables, but use set_value in order
   // not to write any equivalent differential statement to the Adept stack
   radiance[0].set_value(r[0]);
   radiance[1].set_value(r[1]);

   // Loop over the two radiances and add the differential statements to the Adept stack
   for (int i = 0; i < 2; ++i) {
     // Add the first term on the right-hand-side of Equation 1 in the text
     radiance[i].add_derivative_dependence(surface_temperature, dr_dst[i]);
     // Now append the second term on the right-hand-side of Equation 1. The third argument
     // "n" of the following function says that there are n terms to be summed, and the fourth 
     // argument "2" says to take only every second element of the Jacobian dr_dt, since the 
     // derivatives with respect to the two radiances have been interlaced.  If the fourth 
     // argument is omitted then relevant Jacobian elements will be assumed to be contiguous
     // in memory.
     radiance[i].append_derivative_dependence(temperature, &dr_dt[i], n, 2);
   }
 }
\end{lstlisting}
%
In this example, the form of \code{add\_derivative\_dependence} for
one variable on the right-hand-side of the derivative expression has
been used, and the form of \code{append\_derivative\_dependence} for
an array of variables on the right-hand-side has been used. As
described in section \ref{sec:adouble}, both functions have forms that
take single variables and arrays as arguments. Note also that the use
of \code{std::vector<double>} rather than \code{new double[n]} ensures
that if \code{simulate\_radiances} throws an exception, the memory
allocated to hold \code{dr\_dt} will be freed correctly.

\section{Member functions of the \codestyle{Stack} class}
\label{sec:stack}
This section describes the user-oriented member functions of the
\code{Stack} class. Some functions have arguments with default values;
if these arguments are omitted then the default values will be used.
Some of these functions throw \Adept\ exceptions, defined in section
\ref{sec:exceptions}.

\begin{description}
\citem{Stack(bool activate\_immediately = true)} The constructor for the
\codebf{Stack} class.  Normally \codebf{Stack} objects are constructed
with no arguments, which means that the object will attempt to make
itself the currently active stack by placing a pointer to itself into
a global variable.  If another \codebf{Stack} object is currently
active, then the present one will be fully constructed, left in the
unactivated state, and an \code{stack\_already\_active} exception
will be thrown.  If a \codebf{Stack} object is constructed with an
argument ``\codebf{false}'', it will be started in an unactivated
state, and a subsequent call to its member function \codebf{activate}
will be needed to use it.
%
\citem{void new\_recording()} Clears all the information on the stack
in order that a new recording can be started. Specifically this
function clears all the differential statements, the list of
independent and dependent variables (used in computing Jacobian
matrices) and the list of gradients used by the
\codebf{compute\_tangent\_linear} and \codebf{compute\_adjoint} functions.
Note that this function leaves the memory allocated to reduce the
overhead of reallocation in the new recordings.
%
\citem{bool pause\_recording()} Stops recording differential
  information every time an \code{adouble} statement is
  executed. This is useful if within a single program an algorithm
  needs to be run both with and without automatic
  differentiation. This option is only effective within compilation
  units compiled with \code{ADEPT\_RECORDING\_PAUSABLE} defined; if it is,
  the function returns \code{true}, otherwise it returns
  \code{false}. Further information on using this and the following
  function are provided in section \ref{sec:pausable}.
%
\citem{bool continue\_recording()} Instruct a stack that may have
previously been put in a paused state to now continue recording
differential information as normal.  This option is only effective within
compilation units compiled with \code{ADEPT\_RECORDING\_PAUSABLE}
defined; if it is, the function returns \code{true}, otherwise it
returns \code{false}.
%
\citem{bool is\_recording()} Returns \code{false} if recording has
  been paused with \code{pause\_recording()} and the code has been
  compiled with \code{ADEPT\_RECORDING\_PAUSABLE} defined.
  Otherwise returns \code{true}.
%
\citem{void compute\_tangent\_linear()} Perform a tangent-linear
calculation (forward-mode differentiation) using the stored
differential statements.  Before calling this function you need call
the \code{adouble::set\_gradient} or \code{set\_gradients} function (see
section \ref{sec:adouble}) on the independent variables to set the
initial gradients, otherwise the function will throw a
\code{gradients\_not\_initialized} exception. This function is
synonymous with \codebf{forward()}.
%
\citem{void compute\_adjoint()} Perform an adjoint calculation
(reverse-mode differentiation) using the stored differential
statements.  Before calling this function you need call the
\code{adouble::set\_gradient} or \code{set\_gradients} function on the
dependent variables to set the initial gradients, otherwise the
function will throw a \code{gradients\_not\_initialized}
exception. This function is synonymous with \codebf{reverse()}.
%
\citem{void independent(const adouble\&\ x)} Before computing Jacobian
  matrices, you need to identify the independent and dependent
  variables, which correspond to the columns and rows of he Jacobian,
  respectively. This function adds \codebf{x} to the list of
  independent variables. If it is the $n$th variable identified in
  this way, the $n$th column of the Jacobian will correspond to
  derivatives with respect to \codebf{x}.
\citem{void dependent(const adouble\&\ y)} Add \codebf{y} to the
  list of dependent variables.  If it is the $m$th variable identified
  in this way, the $m$th row of the Jacobian will correspond to
  derivatives of \codebf{y} with respect to each of the independent
  variables.
\citem{void independent(const adouble* x\_ptr, \Offset\ n)} Add
  \codebf{n} independent variables to the list, which must be
  stored consecutively in memory starting at the memory pointed to by
  \codebf{x\_ptr}.
\citem{void dependent(const adouble* y\_ptr, \Offset\ n)} Add
\codebf{n} dependent variables to the list, which must be stored
consecutively in memory starting at the memory pointed to by
\codebf{y\_ptr}.
%
\citem{void jacobian(double* jacobian\_out)} Compute the Jacobian matrix, i.e., the gradient of the $m$
dependent variables (identified with the \codebf{dependent(...)}
function) with respect to the $n$ independent variables (identified
with \codebf{independent(...)}. The result is returned in the memory
pointed to by \codebf{jacobian\_out}, which must have been allocated
to hold $m\times n$ values. The result is stored in
column-major order, i.e., the $m$ diemension of the matrix varies
fastest. If no dependents or independents have been identified,
then the function will throw a
\code{dependents\_or\_independents\_not\_identified} exception. In
practice, this function calls \codebf{jacobian\_forward} if $n\le
m$ and \codebf{jacobian\_reverse} if $n>m$.
%
\citem{void jacobian\_forward(double* jacobian\_out)} Compute the Jacobian matrix by executing
$n$ forward passes through the stored list of differential statements;
this is typically faster than \codebf{jacobian\_reverse} for $n\le
m$.
%
\citem{void jacobian\_reverse(double* jacobian\_out)} Compute the Jacobian matrix by executing
$m$ reverse passes through the stored list of differential statements;
this is typically faster than \codebf{jacobian\_forward} for
$n>m$.
%
\citem{void clear\_gradients()} Clear the gradients set with the
\code{set\_gradient} member function of the \code{adouble} class. This
enables multiple adjoint and/or tangent-linear calculations to be
performed with the same recording.
%
\citem{void clear\_independents()} Clear the list of independent
variables, enabling a new Jacobian matrix to be computed from the same
recording but for a different set of independent variables.
%
\citem{void clear\_dependents()} Clear the list of dependent
variables, enabling a new Jacobian matrix to be computed from the same
recording but for a different set of dependent variables.
%
\citem{\Offset\ n\_independents()} Return the number of independent
variables that have been identified.
%
\citem{\Offset\ n\_dependents()} Return the number of dependent
variables that have been identified.
%
\citem{\Offset\ n\_statements()} Return the number of differential
statements in the recording.
%
\citem{\Offset\ n\_operations()} Return the total number of operations
in the recording, i.e the total number of terms on the right-hand-side
of all the differential statements.
%
\citem{\Offset\ max\_gradients()} Return the number of working gradients
that need to be stored in order to perform a forward or reverse pass.
%
\citem{size\_t memory()} Return the number of bytes currently
used to store the differential statements and the working
gradients. Note that this does not include memory allocated but not
currently used.
%
\citem{\Offset\ n\_gradients\_registered()} Each time an
\code{adouble} object is created, it is allocated a unique index that
is used to identify its gradient in the recorded differential
statements. When the object is destructed, its index is freed for
reuse. This function returns the number of gradients currently
registered, equal to the number of \code{adouble} objects currently
created.
%
\citem{void print\_status(std::ostream\&\ os = std::cout)} Print the
current status of the \codebf{Stack} object, such as number of
statements and operations stored and allocated, to the stream
specified by \codebf{os}, or standard output if this function is
called with no arguments.  Sending the \codebf{Stack} object to the
stream using the ``\code{<<}'' operator results in the same behaviour.
%
\citem{void print\_statements(std::ostream\&\ os = std::cout)} Print
the list of differential statements to the specified stream (or
standard output if not specified). Each line corresponds to a separate
statement, for example ``\code{d[3] = 1.2*d[1] + 3.4*d[2]}''.
%
\citem{bool print\_gradients(std::ostream\&\ os = std::cout)} Print
the vector of gradients to the specified stream (or standard output if
not specified). This function returns
\code{false} if no \code{set\_gradient}
function has been called to set the first gradient and initialize the
vector, and \code{true} otherwise. To diagnose what
\codebf{compute\_tangent\_linear} and 
\codebf{compute\_adjoint} are doing, it can be useful to call
\codebf{print\_gradients} immediately before and after.
%
\citem{void activate()} Activate the \codebf{Stack} object by copying
its \code{this} pointer to a global variable that will be accessed by
subsequent operations involving \code{adouble} objects.  If another
\codebf{Stack} is already active, a \code{stack\_already\_active}
exception will be thrown. To check whether this is the case before
calling \codebf{activate()}, check that the \code{active\_stack()}
function (described below) returns \code{0}.
%
\citem{void deactivate()} Deactivate the \codebf{Stack} object by
checking whether the global variable holding the pointer to the
currently active \codebf{Stack} is equal to \code{this}, and if it is,
setting it to \code{0}.
%
\citem{bool is\_active()} Returns \code{true} if the \codebf{Stack}
object is the currently active one, \code{false} otherwise.
%
\citem{void start()} This function was present in version 0.9 to
activate a \codebf{Stack} object, since in that version they were not
constructed in an activated state.  This function has now been
deprecated and will always throw a \code{feature\_not\_available}
exception.
\citem{int max\_jacobian\_threads()} Return the maximum number of
OpenMP threads available for Jacobian calculations.  The number will
be 1 if either the library was or the current source code is compiled
without OpenMP support (i.e.\ without the \code{-fopenmp} compiler and
linker flag). (Introduced in \Adept\ version 1.1.) 
\citem{int set\_max\_jacobian\_threads(int n)} Set the maximum number of
threads to be used in Jacobian calculations to \code{n}, if
possible. A value of 1 indicates that OpenMP will not be used, while a
value of 0 indicates that the maximum available will be used. Returns
the maximum that will be used, which may be fewer than requested,
e.g. 1 if the \Adept\ library was compiled without OpenMP
support. (Introduced in \Adept\ version 1.1.) 
\citem{void preallocate\_statements(int n)} If you know in advance
roughly how many differential statements will be stored by an
algorithm then you may be able to speed-up the first use of the stack
by preallocating the memory needed to store them.  More memory will
still be allocated if needed, but this should reduce the number of
allocations and copies.
\citem{void preallocate\_operations(int n)} Likewise, if you know in
advance roughly how many operations will be stored then you can
speed-up the first use of the stack with this member function.
\end{description}

\noindent The following non-member functions are provided in the
\code{adept} namespace:
\begin{description}
\citem{adept::Stack* active\_stack()} Returns a pointer to the
currently active \codebf{Stack} object, or \code{0} if there is none.
\citem{bool is\_thread\_unsafe()} Returns \code{true} if your code has
been compiled with \code{ADEPT\_STACK\_THREAD\_UNSAFE}, \code{false}
otherwise.
%
\end{description}


\section{Member functions of the \codestyle{adouble} object}
\label{sec:adouble}
This section describes the user-oriented member functions of the
\code{adouble} class. Some functions have arguments with default
values; if these arguments are omitted then the default values will be
used. Some of these functions throw \Adept\ exceptions, defined in
section \ref{sec:exceptions}.
\begin{description}
\citem{double value()} Return the underlying \code{double} value.
%
\citem{void set\_value(double x)} Set the value of the \codebf{adouble}
object to \codebf{x}, without storing the equivalent differential
statement in the currently active stack.
%
\citem{void set\_gradient(const double\&\ gradient)} Set the
gradient corresponding to this \codebf{adouble} variable. The first call
of this function (for any \codebf{adouble} variable) after a new
recording is made also initializes the vector of working gradients.
This function should be called for one or more \codebf{adouble} objects
after a recording has been made but before a call to
\code{Stack::compute\_tangent\_linear()} or
\code{Stack::compute\_adjoint()}.
%
\citem{void get\_gradient(double\&\ gradient)} Set \codebf{gradient}
to the value of the gradient corresponding to this \codebf{adouble}
object. This function is used to extract the result after a call to
\code{Stack::compute\_tangent\_linear()} or
\code{Stack::compute\_adjoint()}. If the \codebf{set\_gradient} function
was not called since the last recording was made, this function will
throw a \code{gradients\_not\_initialized} exception.  The function
can also throw a \code{gradient\_out\_of\_range} exception if new
\codebf{adouble} objects were created since the first
\codebf{set\_gradient} function was called.
%
\citem{void add\_derivative\_dependence(const adouble\&\ r, const
  double\&\ g)} Add a differential statement to the currently active
stack of the form $\delta \codebf{l}=\codebf{g}\times\delta
\codebf{r}$, where \codebf{l} is the \codebf{adouble} object from which
this function is called.  This function is needed to interface to
software containing hand-coded Jacobians, as described in section
\ref{sec:interfacehandcoded}; in this case \codebf{g} is the gradient
$\partial\codebf{l}/\partial\codebf{r}$ obtained from such software.
%
\citem{void append\_derivative\_dependence(const adouble\&\ r, const
  double\&\ g)} Assuming that the same \codebf{adouble} object has just
had its \codebf{add\_derivative\_dependence} member function called,
this function appends ${}+\codebf{g}\times\delta\codebf{r}$ to the
most recent differential statement on the stack.  If the calling
\codebf{adouble} object is different, then a \code{wrong\_gradient}
exception will be thrown. Note that multiple
\codebf{append\_derivative\_dependence} calls can be made in succession.
%
\item[\begin{minipage}{\textwidth}\codesize\texttt{void 
add\_derivative\_dependence(const adouble* r, const double* g,}\\ 
\mbox{ }\texttt{\hspace{18em}\Offset\ n = 1, \Offset\
      m\_stride = 1)}\end{minipage}]
%
Add a differential statement to the currently active stack of the form
$\delta\codebf{l}=\sum_{i=0}^{\codebf{n}-1}\codebf{m[}i\codebf{]}
\times\delta\codebf{r[}i\codebf{]}$, where \codebf{l} is the \codebf{adouble}
object from which this function is called. If the \codebf{g\_stride}
argument is provided, then the index to the \codebf{g} array will be
$i\times\codebf{g\_stride}$ rather than $i$.  This is useful if the
Jacobian provided is oriented such that the relevant gradients for
\codebf{l} are not spaced consecutively.
%
\item[\begin{minipage}{\textwidth}\codesize\texttt{void 
append\_derivative\_dependence(const adouble* rhs, const double* g,}\\ 
\mbox{ }\texttt{\hspace{20em}\Offset\ n = 1, \Offset\
      g\_stride = 1)}\end{minipage}]
%
Assuming that the same \codebf{adouble} object has just called the
\codebf{add\_derivative\_dependence} function, this function appends
${}+\sum_{i=0}^{\codebf{n}-1}\codebf{m[}i\codebf{]}
\times\delta\codebf{r[}i\codebf{]}$ to the most recent differential
statement on the stack. If the calling \codebf{adouble} object is
different, then a \code{wrong\_gradient} exception will be
thrown. The \codebf{g\_stride} argument behaves the same way as in the
previous function described.
\end{description}

\noindent The following non-member functions are provided in the
\code{adept} namespace:
\begin{description}
\citem{double value(const adouble\& x)} Returns the underlying
value of \codebf{x} as a \codebf{double}. This is useful to enable
\codebf{x} to be used in \code{fprintf} function calls. It is
generally better to use \codebf{adept::value(x)} rather than
\codebf{x.value()}, because the former also works if you compile the
code with the \code{ADEPT\_NO\_AUTOMATIC\_DIFFERENTIATION} flag set,
as discussed in section \ref{sec:multipleobjects}.
%
\citem{void set\_values(adouble* x, \Offset\ n, const double* x\_val)}
Set the value of the \codebf{n} \codebf{adouble} objects starting at
\codebf{x} to the values in \codebf{x\_val}, without storing the
equivalent differential statement in the currently active stack.
%
\citem{void set\_gradients(adouble* x, size\_t n, const double*
  gradients)} Set the gradients corresponding to the \codebf{n}
\codebf{adouble} objects starting at \codebf{x} to the \codebf{n}
\code{double}s starting at \codebf{gradients}.  This has the same
effect as calling the \codebf{set\_gradient} member function of each
\codebf{adouble} object in turn, but is more concise.
%
\citem{void get\_gradients(const adouble* y, size\_t n, double*
  gradients)} Copy the gradient of the \codebf{n} \codebf{adouble}
objects starting at \codebf{y} into the \codebf{n} \code{double}s
starting at \codebf{gradients}. This has the same effect as calling
the \codebf{get\_gradient} member function of each \codebf{adouble} object
in turn, but is more concise.  This function can throw a
\code{gradient\_out\_of\_range} exception if new \codebf{adouble}
objects were created since the first \codebf{set\_gradients} function
or \codebf{set\_gradient} member function was called.
\end{description}

\chapter{Using \Adept's array functionality}
\label{chap:arrays}

\section{Introduction}
\label{sec:array_functionality}
The design of \Adept's array capability and many of the functions is
inspired to a significant extent by the built-in array support in
Fortran 90 (and later), and a lesser extent by Matlab, although
implemented in the ``C++ way'', e.g.\ default row-major order with all
array indices starting from zero.  Future additions to the array
capability in \Adept\ will attempt to reproduce built-in Fortran array
functions if available\footnote{This decision may puzzle some readers,
  since Fortran is a dirty word to many C++ users due to the
  limitations of the FORTRAN 77 language. Many of these limitations
  were overcome in Fortran 90, whose array functionality in particular
  is rather well designed. Indeed, the pioneering ``Blitz++'' C++
  array library \cite[]{Veldhuizen1995} also reproduces many Fortran
  array functions. All references to Fortran in this document imply
  the 1990 (or later) standard.}. This design makes \Adept\ a good
choice if you have Fortran code that you wish to convert to C++.
\Adept\ provides the following array functionality:
%
\begin{description}
\item[Multi-dimensional arrays.]  Standard dynamically sized arrays
  can have an arbitrary number of dimensions (although indexing and
  slicing is supported only up to 7), and may refer to non-contiguous
  areas of memory. See section \ref{sec:array}.
\item[Mathematical operators and functions.] \Adept\ supports array
  expressions containing the standard mathematical operators \code{+},
  \code{-}, \code{*} and \code{/}, as well as their assignment
  versions \code{+=}, \code{-=}, \code{*=} and \code{/=}. When applied
  to arrays, they work ``element-wise'', applying the same operation
  to every element of the arrays. \Adept\ also supports array
  operations on all the mathematical functions listed in section
  \ref{sec:ad_functionality}. The following operators and functions
  return boolean array expressions: \code{==}, \code{!=}, \code{>},
  \code{<}, \code{>=} and \code{<=}, \code{isfinite}, \code{isinf} and
  \code{isnan}.  See section \ref{sec:operators}.
\item[Array slicing.] There are many ways to produce an array that
  references a subset of another array, and therefore can be used as
  an lvalue in a statement. Arrays can be indexed with scalar
  integers, a contiguous range of integers, a strided range of
  integers or an arbitrary list of integers.  This is facilitated with
  ``\code{\_\_}'' (a double underscore) and ``\code{end}'', such that
  \code{A(\_\_,end-1)} returns a vector pointing to the penultimate
  column of matrix \code{A}. The member function \code{subset}
  produces an array pointing to a contiguous subset of the original
  array, while \code{diag\_vector} and \code{diag\_matrix} produce
  arrays pointing to the diagonal of the original array.  \code{T}
  produces an array pointing to the transpose of the original array.
  See section \ref{sec:slice}.
\item[Passing arrays to and from functions.] \Adept\ uses a
  reference-counting approach to implement the storage of array data,
  enabling multiple array objects to point to the same data, or parts
  of it in the case of array slices. This makes it straightforward to
  pass arrays to and from functions without having to perform a deep
  copy. See section \ref{sec:passing}.
\item[Array reduction operations.] The functions \code{sum},
  \code{mean}, \code{product}, \code{minval}, \code{maxval} and
  \code{norm2} perform reduction operations that return an array of
  lower rank to the expression they are applied to. The functions
  \code{all} and \code{any} do the same but for boolean
  expressions. \code{count} returns the number of \code{true} elements
  in a boolean expression.
% The function
%  \code{find(A)} returns indices to the \code{true} elements of
%  \code{A}. 
  See section \ref{sec:reduce}.
\item[Array expansion operations.] The functions \code{outer\_product}
  and \code{spread} return an expression of a higher rank than the
  expression they are applied to. See section \ref{sec:expand}
\item[Conditional operations.] Two convenient ways are provided to
  perform an operation on an array depending on the result of a
  boolean expression: \code{where} and \code{find}. The statement
  \code{A.where(B>0)=C} assigns elements of \code{C} to elements of
  \code{A} whenever the corresponding element of \code{B} is greater
  than zero. For vectors only, the same result could be obtained with
  \code{A(find(B>0))=C(find(B>0))}. See section \ref{sec:conditional}.
\item[Fixed-size arrays.] \Adept\ provides a fixed-size array class
  with dimensions (up to seven) that are known at compile time. The
  functionality is very similar to standard dynamic arrays.
\item[Special square matrices.] \Adept\ uses specific classes for
  symmetric, triangular and band-diagonal matrices, the latter of
  which use compressed storage and include diagonal and tridiagonal
  matrices. Certain operations such as matrix multiplication and solving
  linear equations are optimized especially for these objects. See
  section \ref{sec:square}.
\item[Matrix multiplication.] Matrix multiplication can be applied to
  one- and two-dimensional arrays using the \code{matmul} function, of
  for extra syntactic sugar, the ``\code{**}''
  pseudo-operator. \Adept\ uses whatever BLAS (Basic Linear Algebra
  Subroutines) support is available on your system, including
  optimized versions for symmetric and band-diagonal matrices. See
  section \ref{sec:matmul}.
\item[Linear algebra.] \Adept\ uses the LAPACK library to invert
  matrices and solve linear systems of equations. See section
  \ref{sec:la}.
\item[Array bounds and alias checking.] \Adept\ checks at compile time
  that terms in an array expression accord in rank, and at run time
  that they accord in the size of each dimension. Run-time alias
  checking is performed to determine if any objects on the
  right-hand-side of a statement overlap in memory with the
  left-hand-side of the statement, making a temporary copy of the
  right-hand-side if they do. This can be overridden with the
  \code{noalias} function. See section \ref{sec:bounds}.
\end{description}% 
%


\section{The \codestyle{Array} class}
\label{sec:array}
The bread and butter of array operations is provided by the
\code{Array} class template (in the \code{adept} namespace along with
all other public types and classes), which has the following declaration:
\begin{lstlisting}
 namespace adept {
   template <int Rank, typename Type = Real, bool IsActive = false>
   class Array;
 }
\end{lstlisting}
The first template argument provides the number of dimensions of the
array and may be 1 or greater, although indexing and slicing is only
supported up to 7 dimensions. The second argument is the numerical
type being stored and can be any simple integer or real number,
including \code{bool}. The default type is \code{adept::Real}, which
is the default floating-point type the \Adept\ library has been
compiled to use for computing derivatives, and is usually
\code{double}. The final argument states whether the array is
``active'', i.e.\ whether it participates in the differentiation of an
algorithm.

A number of typedefs are provided for the most common types of array:
\code{Vector}, \code{Matrix}, \code{Array3D} and so on up to
\code{Array7D} provide inactive arrays of type \code{Real} and rank
1--7. The corresponding active types are \code{aVector},
\code{aMatrix}, \code{aArray3D} etc. Arrays of other numeric types
have the pattern \code{boolVector}, \code{intVector},
\code{floatVector}, \code{afloatVector}, and similarly for matrices
and higher dimensional arrays. If you wanted shortcuts for
other types you could do the following:
\begin{lstlisting}
 typedef adept::Array<4,unsigned int> uintArray4D;
 typedef adept::Array<2,long double,true> alongdoubleMatrix; // Active
\end{lstlisting}

An \code{Array} with uninitialized elements can be constructed in
numerous ways:
\begin{lstlisting}
 using namespace adept;
 Vector v;                  // Initialize an empty vector
 Array3D A(3,4,5);          // Initialize a 3x4x5 array (up to 7 arguments possible)
 Matrix M(dimensions(3,4)); // The "dimensions" function takes up to 7 arguments
 Matrix N(M.dimensions());  // Make N the same size as M
\end{lstlisting}
In the remaining code examples it will be assumed that
\code{using namespace adept} has already been called.  When new memory
is needed, the \code{Array} object creates a \code{Storage} object
that contains the memory needed, and stores pointers to both the
\code{Storage} object and the start of the data. By default the data
are accessed in C-style row-major order (i.e.\ the final index
corresponds to the array dimension that varies most rapidly in
memory). However, this is flexible since in addition to storing the
length of each of its $n$ dimensions, a rank-$n$ \code{Array} also
stores $n$ ``offsets'' that define the separation of elements in
memory in each dimension. Thus, a 3-by-4 matrix with row-major storage
would store offsets of (4,1). The same size matrix would use
column-major storage simply by storing offsets of (1,3). To make new
arrays use column-major storage, call the following function:
\begin{lstlisting}
 set_array_row_major_order(false);
\end{lstlisting}
Note that this does not change the storage of any existing
objects. Note also that when array expressions are evaluated, the data
are requested in row-major order, so the use of column-major arrays
will incur a performance penalty.

An \code{Array} may also be constructed such that it immediately
contains data:
\begin{lstlisting}
 Vector v = M(__,0); // Link to a existing array, in this case the first column of M
 Vector v(M(__,0));  // Has exactly the same effect as the previous example
 Matrix N = log(M);  // Initialize with the size and values of a mathematical expression
\end{lstlisting}
It can be seen from the constructors involving \code{Vector}s that an
\code{Array} can be configured to ``link'' to part of an existing
\code{Array}, and modifications to the numbers in one will be seen by
the other. This is a very useful feature as it allows slices of an
array to be passed to functions and modified; see section
\ref{sec:slice}. Note that the array or sub-array being linked to must
be of the same rank, type and activeness as the linking array.
Internally, linking is achieved by both the arrays pointing to the
same \code{Storage} object, which itself contains a reference count of
the number of arrays pointing to it. When an \code{Array} is
destructed the reference count is reduced by one and only if it falls
to zero will the data get deallocated. This ensures that if the
\code{Array} being linked to goes out of scope, the linking
\code{Array} will ``steal'' the data.

You can also make an \code{Array} point to data not held in a
\code{Storage} object, for example in a function whose interface is
only in terms of intrinsic C types:
\begin{lstlisting}
 double my_norm2(int n, double* ptr) {
   Vector x(ptr, dimensions(n)); // Create a Vector pointing to existing data
   return norm2(x);              // Use Adept's L2-norm function
 }
\end{lstlisting}
The \code{Vector} in this example can be used in the same way as any
other array, but relies on the existing data not being deallocated for
the lifetime of the \code{Vector}.

After it has been constructed, an \code{Array} can be resized,
relinked or cleared completely as follows:
\begin{lstlisting}
 M.resize(5,2);            // Works up to 7 dimensions
 M.resize(dimension(5,2)); // As above
 N.resize(M.dimensions()); // Resize N to be the same size as M
 v.link(M(end-1,__));      // Size of v set to that of the argument and link to data
 v >>= M(end-1,__);        // Convenient syntax for linking, similar to Fortran's "->"
 M.clear();                // Returns array to original empty state
\end{lstlisting}
The member functions \code{resize} and \code{clear} unlink from any
existing data, which involves deallocation if no other array is
pointing to the same data. If the \code{link} function, or the
alternative ``\code{>>=}'' syntax, is applied with a non-empty array
on the left-hand-side then the existing data will be quietly cleared
before linking to the new data. Note that if you assign one array to
another (e.g.\ \code{N=M}), then they must be of the same size; if
they are not then you should clear the left-hand-side first. By
default, resized arrays are row-major, unless
\code{set\_array\_row\_major(false)} has been called. To explicitly
specify the ordering, you may use the \code{resize\_row\_major} or
\code{resize\_column\_major} member functions in place of
\code{resize}.

The \code{Array} class implements a number of member functions for
inquiring about its properties:
\begin{description}
\citem{size()} Returns the total number of elements, i.e.\ the product
of the lengths of each of the dimensions.
\citem{dimension(i)} Returns the length of dimension \code{i}.
\citem{offset(i)} Returns the separation in memory of elements along
dimension \code{i}.
\citem{gradient\_index()} For active arrays, returns the gradient
index of the first element of the array, which is always positive; for
inactive arrays it returns a negative number.
\citem{empty()} Returns \code{true} if the array is in the empty
state, or \code{false} otherwise.
\citem{dimensions()} Returns a object listing the extent of each
dimension in the \code{Array}, useful for resizing other arrays.  The
object is actually of type \code{ExpressionSize<int Rank>} (where
\code{Rank} is the rank of the array), a thin wrapper for a simple
  \code{int[Rank]} C-array, although it is rare to need to use it
  explicitly.
\citem{offset()} Returns an object (also of type
\code{ExpressionSize<int Rank>}) describing how array indices are
translated into memory offsets.
\end{description}

An \code{Array} may be filled using the \code{<<} operator for the
first element followed by either the \code{<<} or \code{,} operators
for subsequent elements:
\begin{lstlisting}
 Vector v(4);
 v << 1 << 2 << 3 << 4; // Fill the four elements of v
 v << 1, 2, 3, 4;       // Same behaviour but easier on the eye
 v << 1, 2, 3, 4, 5;    // Error: v has been overfilled
 Matrix M(2,4);
 M << 1, 2, 3, 4,       // Filling of multi-dimensional arrays
      5, 6, 7, 8;       // automatically moves on to next dimension
 M << 1, 2, 3, 4,
      v;                // v treated as a row vector here
\end{lstlisting}
For multidimensional arrays, elements are filled such that the final
dimension ticks over fastest (regardless of whether the array uses
row-major storage internally), and new rows are started when a row is
complete. Moreover, other arrays can be part of the list of elements,
provided that they fit in.  In this context, a rank-1 array is treated
as a row vector. An \code{index\_out\_of\_bounds} exception is thrown
if an array is overfilled, while an \code{empty\_array} exception is
thrown if an attempt is made to fill an empty array.

\cxx11 \begin{leftbar} If you compile your code with C++11 features
  enabled then you can use the ``initializer list'' feature to fill
  arrays using the C-like curly bracket syntax:
\begin{lstlisting}
 Vector v;              // Construct an empty vector
 v = {1, 2, 3};         // Resize to length 3 and fill
 Vector w = {1, 2, 3};  // Construct a vector of length 3 and fill
 w = {4.4, 5.5};        // Underfill leads to remaining elements set to zero (as in C)
 w = {6, 7, 8, 9};      // Overfill leads to size_mismatch exception being thrown
 Matrix M = {{1, 2, 3}, // Multi-dimensional arrays use nested curly brackets;
             {4, 5}};   //  ...underfill again leads to remaining elements set to zero
\end{lstlisting}
Another convenient property of this syntax is that temporary arrays
with explicit values can be used in expressions:
\begin{lstlisting}
 v = w * Vector{3.0, 4.2, 5.1};
\end{lstlisting}
\end{leftbar}

When interfacing with other libraries, direct access to the data is
often required. The \code{Array} class provides the following member
functions:
\begin{description}
\citem{data()} Returns a pointer to the first element in the array,
i.e.\ the element found by indexing all the dimensions of the array
with zero. It is up to the caller to understand the layout of the data
in memory and not to stray outside.  Remember that an array may be
strided and the stride may even be negative so that the data returned
from increasing indices are actually from earlier memory
addresses. Note that a double-precision active array is not stored as
an array of \code{adouble} objects, but as an array of \code{double}
data and a single gradient index for the first element. Thus the
pointer returned by \code{data()} will point to the underlying
inactive data.  In contexts where the \code{Array} object is
\code{const}, a \code{const} pointer will be returned. Note that in a
multi-dimensional array, successive array dimensions are not
guaranteed to be contiguous in memory since it is sometimes
advantageous for vectorization for \Adept\ to pad the rows to an alignment
boundary. You can use the output of the \code{offset()} member
function to determine the spacing of the elements in each dimension.
%
\citem{const\_data()} It is sometimes convenient to specify explicitly
that read-only access is required, in which case you can use
\code{const\_data()} to return a \code{const} pointer to the first
element in the array.
\end{description} 

\section{Operators and mathematical functions}
\label{sec:operators}
The operators and mathematical functions listed in section
\ref{sec:ad_functionality} have been overloaded so that they work exactly as you
would expect. Consider this example:
\begin{lstlisting}
 floatVector a(5);      // Inactive single-precision vector
 aVector b(5), c(5);    // Active vectors
 aReal d;               // An active scalar
 // ... other code manipulating a-d ...
 b = 2.0;               // Set all elements of b to a scalar value
 c += 5.0*a + sin(b)/d; // Add the right-hand-side to c
\end{lstlisting}
The penultimate illustrates that all elements of an \code{Array} can
be set to the same value, although note that this will only work if
the array is not in the empty state. The final line illustrates how
terms with different rank, type and activeness can participate in the
same expression. Scalars and arrays can participate in the same
expression on the right-hand-side of a statement provided that the
arrays have the same size as the array on the left-hand-side. Objects
of different type (in this case single and double precision) can be
combined in a mathematical operation, and the type of that operation
will be the larger (higher precision) of the two types. If active and
inactive objects participate in an expression then the left-hand-side
must also be active. Expression templates ensure that no temporary
arrays need to be created to store the output of intermediate parts of
the expression.  The functions \code{max} and \code{min} behave just
like binary operators (such as \code{+} and \code{*}) in this regard,
as shown by the following:
\begin{lstlisting}
 c = max(a,b);          // Element-wise comparison of a and b 
 c = min(a,3.0);        // Return minimum of each element of a and 3
\end{lstlisting}

The examples so far have floating-point results, but some operators
(e.g.\ ``\code{==}'') and some functions (e.g.\ \code{isinf}) take
floating-point arguments and return a boolean.  The \Adept\ versions
take floating-point array expressions as arguments and return
\code{bool} expressions of the same rank and size. Finally, the
\Adept\ versions of the operators \code{!}, \code{||} and \code{\&\&}
take a \code{bool} expression as arguments and return a \code{bool}
expression of the same size and rank.

\section{Array slicing}
\label{sec:slice}
This section concerns the many ways that sub-parts of an \code{Array}
can be extracted to produce an object that can be used as an lvalue;
that is, if the object is modified then it will modify part of the
original \code{Array}. It should be stressed that none of these
methods results in any rearrangement of data in memory, so they should
be efficient.

The first way this can be done is via the function-call and
member-access operators (i.e.\ \code{operator()} and
\code{operator[]}, respectively) of the \code{Array}. In the case of
the function-call operator, the same number of arguments as the rank
of the array must be provided, where each argument states how its
corresponding dimension should be treated.  The nature of the
resulting object depends on the type of all of the arguments in a way
that is similar to how Fortran arrays behave, although note that
array indices always start at 0. The four different behaviours are as
follows:


\begin{description}
\item[Extract single value.] If every argument is an integer scalar or
  scalar expression, then a reference to a single element of the array
  will be extracted. If an argument is an integer expression
  containing \code{end}, then \code{end} will be interpretted to be
  the index to the final element of that dimension (a feature borrowed
  from Matlab). If the array is active then the returned object will
  be of a special ``active reference'' type that can be used as an
  lvalue and ensures that any expressions making use of this element
  can be differentiated. Now for some examples:
  \begin{lstlisting}
 aMatrix A(4,3);
 aReal x = A(1,1);  // Copy element at second row and second column into x
 A(end-1,1) *= 2.0; // Double the element in the penultimate column and 2nd row of A
 A(3) = 4.0;        // Error: number of indices does not match number of dimensions
  \end{lstlisting}
\item[Extract regular subarray.] If every argument is either (i) an
  integer scalar or scalar expression, or (ii) a regular range of
  indices, and there is at least one of (ii), then an \code{Array}
  object will be returned of the same type and activeness as the
  original. However, for each argument of type (i), the rank of the
  returned array will be one less than that of the original. There are
  three ways to express a regular range of indices: ``\code{\_\_}''
  represents all indices of a particular dimension, \code{range(a,b)}
  represents a contiguous range of indices between \code{a} and
  \code{b} (equivalent to \code{a:b} in Fortran and Matlab), and
  \code{stride(a,b,c)} represents a regular range of indices between
  \code{a} and \code{b} with spacing \code{c} (equivalent to
  \code{a:b:c} in Fortran and \code{a:c:b} in Matlab). Note that
  \code{a}, \code{b} and \code{c} may be scalar expressions containing
  \code{end}, but \code{c} must not be zero although it can be
  negative to indicate a reversed ordering. The rank of the returned
  array is known at compile time; thus if range arguments are found at
  run-time to contain only one element (e.g.\ \code{range(1,1)}) then
  the dimension being referred to will be not be removed in the
  returned array but will remain as a singleton dimension. This
  behaviour is the same as indexing an array dimension with \code{1:1}
  in Fortran. Now for some examples:
\begin{lstlisting}
 v(range(1,end-1))           // Subset of vector v that excludes 1st & last points
 A(0,stride(end,0,-1))       // First row of A as a vector treated in reverse order
 A(range(0,0),stride(0,0,1)) // A 1-by-1 matrix containing the first element of A
\end{lstlisting}
\item[Extract irregular subarray.] If an array is indexed as in either
  of the two methods above, except that one or more dimensions is
  instead indexed using a rank-1 \code{Array} of integers, then the
  result is a special ``indexed-array'' type that stores how each
  dimension is indexed.  If it then participates either on the left-
  or right-hand-side of a mathematical expression then when an element
  is requested, the indices will be queried to map the request to
  obtain the correct element from the original array. This is much
  less efficient than using regular ranges of indices as above. It
  also means that if an indexed array is passed to a function
  expecting an object of type \code{Array}, then it will first be
  converted to an \code{Array} and any modifications performed within
  the function will not be passed back to the original array. For
  example:
\begin{lstlisting}
 intVector index(3);
 index << 2, 3, 5;
 Array A(4,4);
 A(0,index) = 2.0; // Set irregularly spaced elements of the first row of A
\end{lstlisting}
\item[Slice leading dimension.] In C, an element is extracted from a
  two-dimensional array using \code{A[i][j]}, and \code{A[i]} returns
  a pointer to a single row of \code{A}, where \code{i} and \code{j}
  are integers. To enable similar functionality, if \code{A} is an
  \Adept\ matrix then \code{A[i]} indexes the leading dimension by
  integer \code{i} returning an array of rank one less than the
  original. This is equivalent to \code{A(i,\_\_)}. Furthermore,
  \code{A[i][j]} will return an individual element as in C, but it
  should be stressed that \code{A(i,j)} is more efficient since it
  does not involve the creation of intermediate arrays.
\end{description}
%
There are a few other ways to produce lvalues that consist of a subset
or a reordering of an array. They are implemented as member functions
of the \code{Array} class, in order to distinguish from non-member
functions that produce a copy of the data and therefore cannot be
usefully used as lvalues.  For example, \code{A.T()} and
\code{transpose(A)} both return the transpose of matrix \code{A}, but
the former is faster since it does not make a copy of the original
data, while the latter is more flexible since it can be applied to
array expressions (e.g.\ \code{transpose(A*B)}).  The member functions
available are:
\begin{description}
\citem{subset(int ibegin0, int iend0, ...)} This function returns a
contiguous subset of an array as an array of the same rank that points
to the original data. It takes twice as many arguments as the array
has dimensions, with each pair of arguments representing the indices
to the first and last element to include from a particular
dimension. Exactly the same result can be obtained using \code{range}
but the \code{subset} form is more concise. For example, for a matrix
\code{M}, \code{M.subset(1,5,3,10)} is equivalent to
\code{M(range(1,5),range(3,10))}.
%
\citem{T()} This function returns the transpose of a rank-2 array (a
matrix). The returned array points to the same data but with its
dimensions reversed. A compile-time error occurs if this function is
used on an array with rank other than 2.  Currently \Adept\ doesn't
allow the transpose of a rank-1 array (a vector), since vectors are
not intended to have an intrinsic orientation.  When orientation
matters, such as in matrix multiplication, the intended orientation
may be inferred from the context or specified explicitly.
%
\citem{permute(int i0, int i1, ...)} This function is the
generalization of the transpose for multi-dimensional arrays: it
returns an array of the same rank as the original but with the
dimensions rearranged according to the arguments. There must be the
same number of arguments as there are dimensions, and each dimension
(starting at 0) must be provided once only. The returned array is
linked to the original; the permutation is achieved simply by
rearranging the list of dimensions and the list of ``offsets'' (the
separation in memory of elements along each dimension individually).
%
\citem{diag\_matrix()} When this function is applied to a rank-1
\code{Array} of length $n$, it returns an $n$-by-$n$ diagonal matrix
(specifically a \code{DiagMatrix}; see section \ref{sec:square}) that
points to the data from the rank-1 array along its diagonal.
%
\citem{diag\_vector()} When this function is applied to a rank-2
\code{Array} with equally sized dimensions, it returns a rank-1 array
pointing to the data along its diagonals.  An
\code{invalid\_operation} exception is thrown if applied to a
non-square matrix, and a compile-time error if applied to an array of
rank other than 2.
%
\citem{diag\_vector(int i)} When applied to a square rank-2 $n$-by-$n$
\code{Array}, this returns a rank-1 array of length
$n-\mathrm{abs}(i)$ pointing to the $i$th superdiagonal of the square
matrix, or the $-i$th subdiagonal if $i$ is negative. An
\code{invalid\_exception} exception occurs if applied to a non-square
matrix, and a compile-time error if applied to an array of rank other
than 2.
%
\citem{submatrix\_on\_diagonal(int ibegin,int iend)} When applied to a
square rank-2 array, this function returns a square matrix that shares
part of the diagonal of the original matrix.  Thus
\code{A.submatrix\_on\_diagonal(int ibegin,int iend)} is equivalent to
\code{A(range(ibegin,iend),range(ibegin,iend))}. Its purpose is to
provide a subsetting facility for symmetric, triangular and
band-diagonal matrices (see section \ref{sec:square}) for which
general array indexing is not available. If applied to a non-square
matrix, an \code{invalid\_operation} exception will be thrown.
%\citem{upper\_matrix()}
%\citem{lower\_matrix()}
%\citem{band\_matrix<LDiag,UDiag>()}
\citem{reshape(int i0, int i1...)} Only applicable to an \code{Array}
of rank 1, this returns a multi-dimensional array whose dimensions are
given by the arguments to the function.  Between 2 and 7 dimensions
are possible. If the arguments are such that the total size of the
returned array would not match the length of the vector, an
\code{invalid\_dimension} exception is thrown.
\end{description}

\section{Passing arrays to and from functions}
\label{sec:passing}
When writing functions taking array arguments, there are three
different ways to do it depending on the extent to which the function
needs to be able to modify the array.  In the case of constant array
arguments, a constant reference should be used; for example:
\begin{lstlisting}
 Real l3norm(const Vector& v) {     // Function returning the L3-norm of a vector
   return cbrt(sum(v*v*v));
 }
 Vector w(3); w << 1.0, 2.0, 3.0;   // Create a test vector
 Real ans1 = l3norm(w);             // Named vector argument
 Real ans2 = l3norm(w(range(0,1))); // Temporary vector argument
 Real ans3 = l3norm(2.0*w);         // Expression implicitly converted to temporary vector
\end{lstlisting}
This function works with all three types of argument.  The last
example illustrates that when an inactive rank-1 expression is passed
to the function, it is evaluated and the result placed in a temporary
vector that is passed to the function.

At the other extreme, we may wish to create a function that modifies
an array argument, including the possibility of changing its size; for
example:
\begin{lstlisting}
 void resize_and_zero(int n, Vector& v) { // A rather pointless function...
   v.resize(n); v = 0.0;
 }
 Vector w(4);
 resize_and_zero(2,w);                    // Results in w={0.0, 0.0}
 resize_and_zero(2,w(range(0,2)));        // Compile error: argument is temporary
 resize_and_zero(2,2.0*w);                // Compile error: argument is not an lvalue
\end{lstlisting}
In this case, due to the C++ rule that a non-constant reference cannot
bind to a temporary object, the function can only take a
\emph{non-temporary} \code{Vector} as an argument.  This is fair
enough; it would not make sense to resize the subset of an array, or
an expression. However, it is very common to want to pass a subset of
an array to a function and for the function to modify the values of
the array, but not to resize it. In \Adept\ this is achieved as
follows:
\begin{lstlisting}
 void square_in_place(Vector v) {
   v *= v;
 }
 Vector w(3); w << 2.0, 3.0, 5.0;
 square_in_place(w);              // Results in w={4.0, 9.0, 25.0}
 square_in_place(w(range(0,1)));  // Results in w={4.0, 9.0, 5.0}
 square_in_place(2.0*w);          // No effect on w
\end{lstlisting}
Even though the \code{Vector} has been passed by value, the
\code{Vector} copy constructor performs a ``shallow copy'', which
means that little more than the array dimensions and a pointer to the
data are copied. Therefore, in the first two examples above the vector
\code{v} inside the function points to data in \code{w}, and can
therefore modify \code{w}.  By contrast, when an expression is passed
to the function, a new \code{Vector} is created to hold the result of
the expression, and when this is modified inside the function it does
not affect the data in the calling routine.

The fact that \code{Array} copy constructors perform shallow copies
also improves the efficiency of functions that return arrays such as
the following:
\begin{lstlisting}
 Matrix square(const Matrix& in) {
   Matrix out = in*in; // Create an matrix containing the result of in*in
   return out;  
 }
 Matrix A(100,100);    // Allocate memory for "A"
 Matrix B = square(A); // Copy constructor: shallow copy of "out" into "B"
\end{lstlisting}
At the \code{return} statement, matrix \code{out} is received by the
copy constructor of matrix \code{B}, so a shallow copy is
performed. This means that the description of matrix \code{out} is
copied to \code{B}, including a pointer to \code{Storage} object
containing both the data and a count of the number of references to
it; this counter is increased by one. Matrix \code{out} is then
destructed, and the counter is immediately reduced by one. The net
result is that \code{B} has ``stolen'' the data in the matrix from
\code{out} without it having been copied, thus avoiding unnecessary
allocation of memory on the heap followed by copying and deallocation. 

The shallow-copy implementation leads to behaviour that users may not
be expecting. If an array is initialized from another array in either
of the following two ways:
\begin{lstlisting}
 Matrix M(3,4);
 Matrix A(M);   // Call copy constructor
 Matrix B = M;  // Call copy constructor
\end{lstlisting}
then the result is that \code{A}, \code{B} and \code{M} share the same
data, rather than a copy being made.  To make a deep copy, it is
necessary to do the following:
\begin{lstlisting}
 Matrix M(3,4);
 Matrix A;      // Create empty matrix
 A = M;         // Call assignment operator for deep copy
\end{lstlisting}
This is annoying, but the alternative is that there would be no clean
way to pass a subset of an array to a function that then modifies its
values. The same behaviour is implemented in the Blitz++ array class
\cite[]{Veldhuizen1995}.

It should be noted that with the introduction of ``move semantics'' in
the C++11 standard, the it is possible to detect when an array
returned from a function is about to be destructed, and therefore
invoke a move constructor that implements a shallow copy. This negates
one of the two reasons from making the copy constructor execute only a
shallow copy.  But it does not help in passing array subsets to
functions, unless two versions of every function were created, one
accepting an lvalue reference (\code{Array\&}) and the other accepting
an rvalue reference (\code{Array\&\&}), which is hardly practical.

\cxx11 \begin{leftbar}If you compile your code with C++11 features
  enabled then move semantics can sometimes make assignment more
  efficient. Consider code calling the \code{square} function above:
\begin{lstlisting}
 Matrix A(10,10), B(10,10);
 B = square(A); // Move assignment operator performs shallow copy
 Matrix C(B);   // B and C now share the same data
 B = square(A); // Move assignment operator performs deep copy
\end{lstlisting}
  Both assignments are to temporary objects about to be destructed, so
  the move assignment operator is called. This operator checks how
  many references there are to the data in \code{B}. In the first case
  there is only one reference, so the data in \code{B} can safely be
  discarded and a shallow copy (a ``move'') of the data in the
  temporary is performed. In the second case there are two references,
  so a deep copy must be performed in order that \code{C} sees the
  change in \code{B}.
\end{leftbar}

\section{Array reduction operations}
\label{sec:reduce}
A family of functions return a result that is reduced in rank compared
to their argument, and operate in the same way as Fortran functions of
the same name.  Consider the \code{sum} function, which can be used
either to sum all the elements in an array expression and return a
scalar, or to sum elements along the dimension specified in the second
argument and return an array whose rank is one less than the first
argument:
\begin{lstlisting}
 Array A(3,4);
 Real x = sum(A);     // Sum all elements of matrix A
 Vector v = sum(A,1); // Sum along the row dimension returning a vector of length 3
\end{lstlisting}
Functions that are used in the same way are \code{mean},
\code{product}, \code{minval}, \code{maxval} and \code{norm2} (the
square-root of the sum of the squares of each element).  Note the
difference between \code{maxval} and \code{max}: the behaviour of
\code{max} is outlined in section \ref{sec:operators}. Three further
functions operate in the same way but on boolean arrays: \code{all}
returns \code{true} only if all elements are \code{true}, \code{any}
returns \code{true} if any element is \code{true} (and \code{false}
otherwise), while \code{count} returns the number of \code{true}
elements.  Each of these can work on an individual dimension as with
\code{sum} and friends.

A further function, \code{dot\_product(a,b)}, takes two arguments that
must be rank-1 arrays of the same length and returns the dot
product. This is essentially the same as \code{sum(a*b)}.

\section{Array expansion operations}
\label{sec:expand}
The function \code{outer\_product(x,y)} returns the outer product of
two rank-1 expressions; if ${\bf x}$ and ${\bf y}$ are interpreted as
column vectors then ${\bf xy}^T$ is returned. If \code{outer\_product}
is used in an expression then an intermediate matrix object is not
created to store it.

The function \code{spread<dim>(A,n)} returns an array that replicates
the \code{A} array \code{n} times along dimension \code{dim}. The
returned array has a rank one larger than \code{a} whose dimension
\code{dim} is \code{n} and the remaining dimensions are the same as
those of \code{A}. It is essentially the same as the Fortran function
of the same name, but \code{dim} is provided as a template arguent
since performance is improved if this is known at compile time.  The
following illustrates \code{spread} for an argument of rank 1:
\begin{lstlisting}
 Vector v(3); v << 1, 2, 3;
 Matrix M0 = spread<0>(v,2);
 // M1 contains {{1, 2, 3},
 //              {1, 2, 3}}
 Matrix M1 = spread<1>(v,2);
 // M2 contains {{1, 1},
 //              {2, 2},
 //              {3, 3}}
\end{lstlisting}
Note that \code{spread<1>(x,y.size())*spread<0>(y,x.size())} gives the
same result as \code{outer\_product(x,y)}.

\section{Conditional operations}
\label{sec:conditional}
There are two main ways to perform an operation on an array depending
on the result of a boolean expression. The first is similar to the
Fortran \code{where} construct:
\begin{lstlisting}
 Array A(3,4);
 Array B(3,4);
 A.where(B > 0.0)   = 2.0 * B;            // Only assign to A if B > 0
 A.where(!isnan(B)) = either_or(-B, 0.0); // Read from either one expression or the other
\end{lstlisting}
In the first example, \code{A} is only assigned if a condition is met,
and therefore \code{A} must be of the same size and rank of the
boolean expression. In the second example \code{A} is filled with
elements from the first argument of \code{either\_or} if the boolean
expression is \code{true}, or from the second argument otherwise; if
\code{A} is empty then it will be resized to the size of the boolean
expression. In both cases, the expressions on the right-hand-side may
be scalars or array expressions of the same size as the boolean
expression.  Equivalent expressions are possible replacing the
assignment operator with the \code{+=}, \code{-=}, \code{*=} and
\code{/=} operators, in which case \code{A} must already be the same
size as the boolean expression.

An alternative approach that works with only vectors uses the
\code{find} function. This is similar to the equivalent Matlab
function and returns an \code{IndexVector} (a vector of integers of
sufficient precision to index an array) containing indices to the
\code{true} elements of the vector:
\begin{lstlisting}
 Vector v(10), w(10);
 v(find(v > 5.0)) = 3.0;
 IndexVector index = find(v > 5.0);
 v(index) = 2.0 * u(index);
\end{lstlisting}
This will work if no \code{true} elements are found: \code{find} will
return an empty array, and when \code{v} is indexed by an empty
vector, no action will be taken.  In general, \code{find} is less
efficient than \code{where}.

\section{Fixed-size arrays}
\label{sec:fixed}
The size of the \code{Array} class is dynamic, which is somewhat
sub-optimal for small arrays whose dimensions are known at compile
time. \Adept\ provides an alternative class template for an array
whose size is known at compile time and whose data are stored on the
stack. It has the following declaration:
\begin{lstlisting}
 namespace adept {
   template <typename Type, bool IsActive, int Dim0, int Dim1 = 0, ...>
   class FixedArray;
 }
\end{lstlisting}
The type (e.g.\ \code{double}) and activeness are specified by the
first two template arguments, while the remaining template arguments
provide the size of the dimensions, up to 7.  Only as many sizes need
to be specified as there are dimensions.  A user working with arrays
of a particular size could use \code{typedef} to provide convenient
names; for example:
\begin{lstlisting}
 typedef FixedArray<double,false,4>   Vector4;
 typedef FixedArray<double,false,4,4> Matrix44;
 typedef FixedArray<double,true,4>    aVector4;
 typedef FixedArray<double,true,4,4>  aMatrix44;
\end{lstlisting}
In the \code{adept} namespace, \Adept\ defines \code{Vector2},
\code{Vector3}, \code{Matrix22}, \code{Matrix33} and their active
counterparts.

Fixed arrays have all the same capabilities as dynamic arrays, with a
few exceptions:
\begin{itemize}
\item Since their size is fixed, there are no member functions
  \code{resize}, \code{clear} or \code{in\_place\_transpose}.
\item Since for the lifetime of the object it is associated with data
  on the stack, it cannot link to other data.  This means that there
  is no member function \code{link}, and also if it is passed by value
  to a function then the contents of the array will be copied, rather
  than the behaviour of the \code{Array} class where the receiving
  function links to the original data.
\end{itemize}
All the same slicing operations are available as discussed in section
\ref{sec:slice}, and they return the same types when applied to fixed
arrays as they do when applied to dynamic arrays.  Thus most
operations return an \code{Array} object that links to a subset of the
data within the \code{FixedArray} object.

\section{Special square matrices}
\label{sec:square}
\Adept\ offers several special types of square matrix that can
participate in array expressions.  They are more efficient than
\code{Array}s in certain operations such as matrix multiplication and
assignment, but less efficient in operations such as accessing
individual elements. All use an internal storage scheme compatible
with BLAS (Basic Linear Algebra Subprograms).  All are specializations
of the \code{SpecialMatrix} class template, which has the following
declaration:
\begin{lstlisting}
 namespace adept {
   template <typename Type, class Engine, bool IsActive = false>
   class SpecialMatrix;
 }
\end{lstlisting}
The first template argument is the numerical type, the second provides
the functionality specific to the type of matrix being simulated, and
the third states whether the matrix participates in the
differentiation of an algorithm. The specific types of special matrix
are as follows:
\begin{description}
\item[Square matrices.] \code{SquareMatrix} provides a dense square
  matrix of type \code{Real} with \code{aSquareMatrix} its active
  counterpart. Its functionality is similar to a rank-2 \code{Array},
  except that its dimensions are always equal and the data along its
  fastest varying dimension are always contiguous in memory, which may
  make it faster than \code{Array} in some instances.
\item[Symmetric matrices.] \code{SymmMatrix} provides a symmetric
  matrix of type \code{Real}, and \code{aSymmMatrix} is its active
  equivalent. Internally this type uses row-major unpacked storage
  with the data held in the lower triangle of the array and zeros in
  the upper triangle (equivalent to column-major storage with data in
  the upper triangle). If the oposite configuration is required then
  it is available by specifying different template arguments to the
  \code{SpecialMatrix} class template.  Note that with normal access
  methods, the storage scheme is opaque to the user; for example,
  \code{S(1,2)=2.0} and \code{S(2,1)=2.0} have the same effect.
\item[Triangular matrices.] \code{LowerMatrix} and \code{UpperMatrix}
  (and their active equivalents prefixed by ``\code{a}'') provide
  triangular matrices of type \code{Real}. Internally they use
  row-major unpacked storage, although column-major storage is
  available by specifying different template arguments to the
  \code{SpecialMatrix} class template.
\item[Band diagonal matrices.] \code{DiagMatrix}, \code{TridiagMatrix}
  and \code{PentadiagMatrix} provide diagonal, tridiagonal and
  pentadiagonal \code{Real} matrices, respectively (with their active
  equivalents prefixed by ``\code{a}''). Internally, row-major
  BLAS-type band storage is used such that an $n$-by-$n$ tridiagonal
  matrix stores $3n$ rather than $n^2$ elements. \Adept\ supports
  arbitrary numbers of sub-diagonals and super-diagonals, accessible
  by specifying different template arguments to the
  \code{SpecialMatrix} class template.
\end{description}
A \code{SpecialMatrix} can be constructed and resized as for
\code{Array}s (see section \ref{sec:array}), with the following
additions:
\begin{lstlisting}
 SymmMatrix S(4);  // Initialize a 4-by-4 symmetric matrix
 S.resize(5);      // Resize to a 5-by-5 matrix
\end{lstlisting}
These are applicable to all types of \code{SpecialMatrix}.

In terms of array indexing and slicing, the member functions \code{T},
\code{diag} and \code{diag\_submatrix} described in section
\ref{sec:slice} are all available, but if you index a
\code{SpecialMatrix} with \code{S(a,b)} then \code{a} and \code{b} must
be scalars or scalar expressions. For triangular or band-diagonal
matrices, if the requested element is one of the zero parts of the
matrix then it can only be used as an rvalue in an expression. If you
wish to extract arbitrary subarrays from a \code{SpecialMatrix} then it
must first be converted to a \code{Matrix}:
\begin{lstlisting}
 SymmMatrix S(6);
 intVector index(3);
 index << 2, 3, 5;
 Matrix M = Matrix(S)(index,stride(0,4,2));
\end{lstlisting}


\section{Matrix multiplication}
\label{sec:matmul}
Matrix multiplication may be invoked in two equivalent ways: using the
\code{matmul} function or the ``\code{**}'' pseudo-operator. Following
Fortran, the two arguments may be either rank-1 or rank-2, but at
least one argument must be of rank-2. The orientation of any rank-1
argument is inferred from whether it is the first or second argument,
as shown here:
\begin{lstlisting}
 Matrix A(3,5), B(5,3), C;
 Vector v(5), w;
 C = matmul(A,B); // Matrix-matrix multiplication: return a 3x3 matrix
 w = matmul(v,B); // Interpret v as a row vector: return a vector of length 3
 w = matmul(A,v); // Interpret v as a column vector: return a vector of length 3
\end{lstlisting}
In this way it is never necessary to transpose a vector; the
appropriate orientation to use is inferred from the context.  You may
find it clearer to use ``\code{**}'' for matrix multiplication as
illustrated here:\footnote{A drawback of the \code{**} interface with
  the orientation of vector arguments being inferred is that in an
  expression like \code{A**v**B} (where \code{A} and \code{B} are
  matrices and \code{v} is a vector), \code{v} is interpreted as a
  column vector in \code{A**v}, which returns a column vector result,
  but this result is then implicitly transposed when it is used as the
  left-hand argument of the matrix multiplication with \code{B}.
  Moreover, the order of precedence affects the result, since this
  expression will not give the same answer as \code{A**(v**B)}.
 % I may
 % consider introducing additional constraints and features in future
 % versions to require users to more explicitly state what they mean in
 % such situations, to reduce the chance of accidental mistakes.
}
\begin{lstlisting}
 Matrix A(3,5), B;
 SymmMatrix S(5);                // 5-by-5 symmetric matrix
 Vector c, x(5);
 c = A **  log(S) ** x;          // Returns a vector of length 3
 c = matmul(matmul(A,log(S)),x); // Equivalent to the previous line but using matmul
 c = A ** (log(S) ** x);         // As the previous example but more efficient
 B = 2.0 * S ** A.T();           // Returns a 5-by-3 matrix
 B = 2.0 * S ** A;               // Run-time error: inner dimensions don't match
\end{lstlisting}
The ``\code{**}'' pseudo-operator has been implemented in \Adept\ by
overloading the dereference operator such that ``\code{*A}'' returns a
special type when applied to array expressions, and overloading the
multiply operator to perform matrix multiplication when one of these
types is on the right-hand-side. This means that \code{**} has the
same precedence as ordinary multiplication, and both will be applied
in order of left to right.  Thus, in the first example above,
matrix-matrix multiplication is performed followed by matrix-vector
multiplication. The second example shows how to make this more
efficient with parentheses to specify that the rightmost matrix
multiplication should be applied first, leading to two matrix-vector
multiplications.  The final example shows an expression that would
fail at runtime with an \code{inner\_dimension\_mismatch} exception
due to the matrix multiplication being applied to matrices whose inner
dimensions do not match.

You cannot use \code{matmul} or ``\code{**}'' for vector-vector
multiplication, since it is ambiguous whether you require the inner
product (dot product) or the outer product. Therefore you must
explicitly call the function \code{dot\_product} (section
\ref{sec:reduce}) or \code{outer\_product} (section \ref{sec:expand}).

In order to get the best performance, \Adept\ does not use expression
templates for matrix multiplication but rather calls the appropriate
level-2 BLAS function for matrix-vector multiplication and level-3
BLAS function for matrix-matrix multiplication. For matrix
multiplication involving active vectors and matrices, \Adept\ first
uses BLAS to perform the matrix multiplication and then stores the
equivalent differential statements. There are therefore a few factors
that users should be aware of in order to get the best performance:
\begin{itemize}
\item If an array expression rather than an array is provided as an
  argument to matrix multiplication, it will first be converted to an
  \code{Array} of the same rank. Therefore, if the same expression is
  used more than once in a sequence of matrix multiplications, better
  performance will be obtained by precomputing the array expression
  and storing it in a temporary matrix:
\begin{lstlisting}
 Matrix A(5,5), B(5,5), C(5,5), D(5,5)
 // Slow implementation:
 C = transpose(2.0*A*B) ** (2.0*A*B);
 D = (2.0*A*B) ** C;
 // Faster implementation:
 {
   Matrix tmp = 2.0*A*B;
   C = tmp.T() ** tmp;
   D = tmp ** C;
 } // "tmp" goes out of scope here
\end{lstlisting}
\item If the left-hand argument of a matrix multiplication is a
  symmetric, triangular or band matrix then a specialist BLAS function
  will be used that is faster than the one for general dense
  matrices. \Adept\ may not be able to tell if the result of an array
  expression is symmetric, triangular or has a band structure, and so
  may not call the most efficient BLAS function. The user can help as
  follows:
\begin{lstlisting}
 SymmetricMatrix S(5,5)
 Matrix A(5,5), B(5,5)
 B = (2.0*exp(S)) ** A;           // Slower
 B = SymmMatrix(2.0*exp(S)) ** A; // Faster
\end{lstlisting}
\item BLAS requires that the fastest-varying dimension of input
  matrices are contiguous and increasing. This is always the case for
  the special square matrices described in section \ref{sec:square},
  but not necessarily for a \code{Matrix} or an \code{aMatrix}, which are
  particular cases of the general \code{Array} type. If the
  fastest-varying dimension of such a matrix is not contiguous and
  increasing then \Adept\ will copy it to a temporary matrix before
  invoking matrix multiplications, as in the following example:
\begin{lstlisting}
 Matrix A(5,5), B, C(5,5);
 B.link(A(__, stride(end,1,-1)); // Fastest varying dim is contiguous but decreasing
 C = A ** A; // Matrix multiplication applied directly with A
 C = B ** B; // Adept will copy B to a temporary matrix before multiplication
\end{lstlisting}
\end{itemize}

An additional member function to mention in this section is
\code{in\_place\_transpose()}, which is only applicable to
matrices. It transposes the matrix by swapping the dimensions and the
offsets to each dimension, but leaving the actual data untouched.
This means that a matrix with row-major storage will be changed to
column-major, and vice versa.

\Adept\ can differentiate expressions involving matrix multiplcation,
but this is far from optimal in \Adept\ version 2.0, for two
reasons. Firstly, only differentiation of dense matrices has been
implemented, so when matrix multiplication is applied to active
``special matrices'' (symmetric, band, upper-triangular and
lower-triangular matrices), they are first copied to a dense
matrix. Secondly, the \Adept\ stack format can currently only store
differential statements for scalar expressions, which for matrix
multiplication leads to lots of repeated values on the stack. A future
version of \Adept\ will redesign the stack to allow matrices to be
stored in it; this will be much faster and much less memory-hungry.

\section{Linear algebra}
\label{sec:la}
\Adept\ provides the functions \code{solve} and \code{inv} to solve
systems of linear equations and to invert a matrix, respectively,
which themselves call the most appropriate function from
LAPACK.
\begin{lstlisting}
 Matrix A(5,5), Ainv(5,5), X(5,5), B(5,5);
 SymmMatrix S(5), Sinv(5);
 Vector x(5), b(5);
 Ainv = inv(A);     // Invert general square matrices using LU decomposition
 Sinv = inv(S);     // Invert symmetric matrices using Cholesky decomposition
 x = solve(A,b);    // Solve general system of linear equations
 X = solve(S,B);    // Solve symmetric system of linear equations with matrix right-hand-side
\end{lstlisting}
\iffalse
As for matrix multiplication described in section \ref{sec:matmul}, if
the arguments to \code{solve} and \code{inv} are not matrices with
fastest-varying dimensions that are contiguous and increasing, then
\Adept\ will first convert them to temporary matrices before
performing the operation.
\fi

Statements involving \code{solve} and \code{inv} cannot yet be
automatically differentiated. When the \Adept\ stack is redesigned to
hold matrices, this capability will be added.

\section{Bounds and alias checking}
\label{sec:bounds}
When encountering an array or active expression, \Adept\ performs
several checks to test the validity of the expression both at compile
time and at runtime:
\begin{description}
\item[Activeness check.] An expression in which an active expression
  is assigned to an inactive array will fail to compile.
\item[Rank check.] An expression will fail to compile if the rank of
  the array on the left-hand-side of the ``\code{=}'' operator (or the
  operators ``\code{+=}'', ``\code{*=}'', etc.) does not match the
  rank of the array expression on the right-hand-side. However, a
  scalar (rank-0) expression can be assigned to an array of any rank;
  its value will be assigned to all elements of the
  array. Compile-time rank checks are also performed for each binary
  operation (binary operators such as ``\code{+}'' and binary
  functions such as \code{pow}) making up an array expression:
  compilation will fail if the two arguments do not have the same rank
  and neither is of rank 0.
\item[Dimension check.] When a binary operation is applied to two
  array expressions of rank $n$ then \Adept\ checks at run-time that
  each of the $n$ dimensions has the same length. Otherwise, a
  \code{size\_mismatch} exception is thrown.
\item[Alias check.] By default, \Adept\ checks to see whether the memory
  referenced in the array object on the left-hand-side of a statement
  overlaps with the memory referenced by any of the objects on the
  right-hand-side, as in this example of a shift-right operation:
\begin{lstlisting}
 Vector v(6);
 v(range(1,end)) = v(range(0,end-1));
\end{lstlisting}
  In order to prevent the right-hand-side changing during the
  operation, \Adept\ copies the expression on the right-hand-side to a
  temporary array and then assigns the left-hand-side array to this
  temporary, which is equivalent to the following:
\begin{lstlisting}
 {
   Vector tmp;
   tmp = v(range(0,end-1));
   v(range(1,end)) = tmp;
 } // tmp goes out of scope here
\end{lstlisting}
  However, for speed \Adept\ does not check to see whether individual
  memory locations are shared; rather the start and end memory
  locations are checked to see if they overlap. This means that for
  certain strided operations, copying to a temporary array is
  unnecessary.  Nor is it necessary if elements of an array will be
  accessed in exactly the same order on the left-hand-side as the
  right-hand-side. If the user is sure that alias checking is not
  necessary then he or she can override alias checking for part or all
  of an array expression using the \code{noalias} function, as
  follows:
\begin{lstlisting}
 v(stride(1,end,2)) = noalias(v(stride(0,end-1,2))); // No overlap between RHS and LHS
 v = 1.0 + noalias(exp(v));                          // LHS & RHS accessed in same order 
\end{lstlisting}
  Note that for speed, alias checking is not performed if the
  left-hand-side is a \code{FixedArray}, since such arrays can never
  point to another location and therefore aliasing is less likely to
  arise. Aliasing is still possible if one of the terms on the
  right-hand-side points to the data in the \code{FixedArray} on the
  left. In this case, you can use the \code{eval} function, which
  takes a non-scalar expression as an argument, and returns an array
  containing a copy of the data. For example:
\begin{lstlisting}
 FixedArray<Real,false,3> v = {1.0, 2.0, 3.0}; // C++11 initialization of inactive vector
 v = v(stride(end,0,-1));                      // Aliasing leads to v = {3.0, 2.0, 3.0}
 v = eval(v(stride(end,0,-1)));                // Expected result:  v = {3.0, 2.0, 1.0}
\end{lstlisting}
  To avoid the overhead of alias checking, you can define the
  preprocessor variable \code{ADEPT\_NO\_ALIAS\_CHECKING}, but then it
  is up to the user to identify the statements where aliasing will
  occur and use the \code{eval} function to ensure the correct
  behaviour.
\item[Bounds check.] If the preprocessor variable
  \code{ADEPT\_BOUNDS\_CHECKING} is defined then additional run-time
  checks will be performed when an array is indexed or sliced using
  the methods described in section \ref{sec:slice}; if an index is
  ount of bounds then a \code{index\_out\_of\_bounds} exception will
  be thrown.  This makes indexing and slicing of arrays slower so
  would normally only be used for debugging.
\end{description}

\section{Automatic differentiation capabilities specific to arrays}
Section \ref{sec:adjoint} described how the \code{get\_gradient()}
member function could be used to extract the gradients from a scalar
\code{adouble} object after applying forward- or reverse-mode
differentiation. In the same way, gradients may be extracted from
active \code{Array} and \code{FixedArray} objects, returning an
inactive \code{Array} of the same rank and size. For example, to
compute the derivative of a \code{norm2} operation, we could do the
following:
\begin{lstlisting}
 Stack stack;                     // Stack to store differential statements
 aVector x = {1.0, 2.0, 3.0};     // C++11 initialization
 stack.new_recording();           // Clear any stored differential statements
 aReal y = norm2(x);              // Perform operation to be differentiated
 y.set_gradient(1.0);             // Seed the independent variable
 stack.reverse();                 // Reverse-mode differentiation
 Vector dy_dx = x.get_gradient(); // Extract vector of derivatives
\end{lstlisting}

\section{Array thread safety}
\label{sec:thread}
There are numerous ways of obtaining an \code{Array} that links to
data in another \code{Array} object; not only the ``\code{>>=}'' link
operator described in section \ref{sec:array}, but also the various
subsetting member functions described in section \ref{sec:slice}, and
even just passing arrays to and from functions. This avoids deep
copying and so improves efficiency. In addition to the new \code{Array}
pointing to the same data, it also points to the same \code{Storage}
object, and when a new link is created, the counter in this object
indicating the number of objects pointing to it is incremented. This
ensures that the data will remain provided there is at least one
object linking to it.  A downside of this model is that if multiple
threads access an array simultaneously, even if just to read it, then
the reference counter can become corrupted.  There are two solutions
to this problem. 

\cxx11 \begin{leftbar} If you are using C++11 then you can define the
  \code{ADEPT\_STORAGE\_THREAD\_SAFE} preprocessor variable, which
  makes the reference counter in \code{Storage} objects of type
  \code{std::atomic<int>} and thereby protects all operations on them
  by a mutex. This may degrade the efficiency of your code since the
  mutex will be redundant in single-threaded code. \end{leftbar}

Alternatively, we use the capability of arrays to access data not held
in a \code{Storage} object. The \code{Array} and \code{SpecialMatrix}
classes have a \code{soft\_link()} member function that returns an
object of the same type, size and activeness, which points to the same
data but does not contain a link to the \code{Storage} object:
\begin{lstlisting}
 Matrix M(2,2);
 // ...enter multi-threaded environment
 Matrix N;
 N >>= M.soft_link();            // N links to same data as M but without Storage object
 Vector v = M.soft_link()(__,0); // v links to subset of M but without Storage object
                                 // (recall that the copy constructor is called here) 
\end{lstlisting}
The linked objects may be used in the same way as any other
\code{Array}. This is demonstrated in the
\code{test\_thread\_safe\_arrays} test program.
\section{Writing an array to a stream}
As you would expect, an array can be written to a stream with the
``\code{<<}'' operator:
\begin{lstlisting}
 Vector v = {1, 2};            // Using C++11 initializer lists
 Matrix M = {{3, 4}, {5, 6}};  // for convenience
 std::cout << v << "\n";
 std::cout << M << "\n";
\end{lstlisting}
which by default produces
\begin{lstlisting}
 1 2
 3 4
 5 6
\end{lstlisting}
You can change the output to use curly brackets to indicate
the dimensions of the array as follows:
\begin{lstlisting}
 Vector v = {1, 2};
 Matrix M = {{3, 4}, {5, 6}};
 adept::set_array_print_style(PRINT_STYLE_CURLY);
 std::cout << "v = " << v << ";\n";
 std::cout << "M = " << M << ";\n";
\end{lstlisting}
which produces output that looks like C/C++ code:
\begin{lstlisting}
 v = {1, 2};
 M =
 {{3, 4},
  {5, 6}};
\end{lstlisting}
The available print styles for use by \code{set\_array\_print\_style}
are \code{PRINT\_STYLE\_PLAIN} (default), \code{PRINT\_STYLE\_CURLY},
\code{PRINT\_STYLE\_CSV} (comma-separated values) and
\code{PRINT\_STYLE\_MATLAB} (matrix ordering indicated by Matlab-style
semi-colons and square brackets).

\chapter{General considerations}
\label{chap:gen}

\section{Setting and checking the global configuration}
\label{sec:settings}
\noindent The following non-member functions are provided in the
\code{adept} namespace:
\begin{description}
\citem{std::string version()} Returns a string containing the version
number of the \Adept\ library (e.g. ``\code{1.9.8}'').
\citem{std::string compiler\_version()} Returns a string containing
the compiler name and version used to compile the \Adept\ library.
\citem{std::string compiler\_flags()} Returns a string containing the
compiler flags used when compiling the \Adept\ library.
\citem{std::string configuration()} Returns a multi-line string
listing numerous aspects of the way \Adept\ has been configured.
\citem{bool have\_matrix\_multiplication()} Returns \code{true} if the
Adept library has been compiled with BLAS support, \code{false}
otherwise.
\citem{bool have\_linear\_algebra()} Returns \code{true} if the
Adept library has been compiled with LAPACK support, \code{false}
otherwise.
\citem{int set\_max\_blas\_threads(int n)} Set the maximum number of
threads used for matrix operations by the BLAS library, or zero to use
the upper limit on your system. The number returned is the number
actually used.  
\citem{int max\_blas\_threads()} Return the maximum number of
threads available for matrix operations by the BLAS library.
%
\end{description}

The preprocessor can detect the \Adept\ version at run-time via the
\code{ADEPT\_VERSION} preprocessor variable, which is an integer
variable with the digits $abbcc$ corresponding to \Adept\ version
$a.bb.cc$. This could be used to activate a different compile path
dependent on the version, or even to fail to compile if the version is
not recent enough:
\begin{lstlisting}
#if ADEPT_VERSION < 10910
#error "Adept >= 1.9.10 is required by this program"
#endif
\end{lstlisting}

\section{Parallelizing \Adept\ programs}
\Adept\ currently has limited built-in support for parallelization. If
the algorithms that you wish to differentiate are individually small
enough to be treated by a single processor core, and you wish to
differentiate multiple algorithms independently (or the same algorithm
but with multiple sets of inputs) then parallelization is
straightforward. This is because the global variable containing a
pointer to the \Adept\ stack uses thread-local storage.  This means
that if a process spawns multiple threads (e.g.\ using OpenMP or
Pthreads) then each thread can declare one \code{adept::Stack} object
and all \code{adouble} operations will result in statements being
stored on the stack object specific to that thread.  The
\Adept\ package contains a test program \code{test\_thread\_safe} that
demonstrates this approach in OpenMP.

If your problem is larger and you wish to use parallelism to speed-up
the differentiation of a single large algorithm then the build-in
support is more limited. Provided your program and the \Adept\ library
were compiled with OpenMP enabled (which is the default for the
\Adept\ library if your compiler supports OpenMP), the computation of
Jacobian matrices will be parallelized.  By default, the maximum
number of concurrent threads will be equal to the number of available
cores, but this can be overridden with the
\code{set\_max\_jacobian\_threads} member function of the \code{Stack}
class.  Note that the opportunity for speed-up depends on the size of
your Jacobian matrix: for an $m\times n$ matrix, the number of
independent passes through the stored data is $\mathrm{min}(m,n)$ and
each thread treats \code{ADEPT\_MULTIPASS\_SIZE} of them (see section
\ref{sec:configuring_lib}), so the maximum number of threads that can
be exploited is $\mathrm{min}(m,n)/$\code{ADEPT\_MULTIPASS\_SIZE}.
Again, the \code{test\_thread\_safe} program can demonstrate the
parallelization of Jacobian calculations.  Note, however, that if the
\code{jacobian} function is called from within an OpenMP thread
(e.g.\ if the program already uses OpenMP with each thread containing
its own \code{adept::Stack} object), then the program is likely not to
be able to spawn more threads to assist with the Jacobian calculation.

If you need Jacobian matrices then the ability to parallelize the
calculation of them is useful since this tends to be more
computationally costly than recording the original algorithm.  If you
only require the tangent-linear or adjoint calculations (equivalent to
a Jacobian calculation with $n=1$ or $m=1$, respectively), then
unfortunately you are stuck with single threading. It is intended that
a future version of \Adept\ will enable all aspects of differentiating
an algorithm to be parallelized with either or both of OpenMP and MPI.

If your BLAS library has support for parallelization then be aware
that the performance may be poor if other parts of the program are
parallelized.  This occurs with OpenBLAS, which uses Pthreads, if you
also use parallelized Jacobian calculations, which use OpenMP.  In
this instance you can turn off parallelization of array operations
with the \code{set\_max\_blas\_threads(1)} function in the
\code{adept} namespace.  The number of available threads for array
operations is returned by the \code{max\_blas\_threads()} function.
Alternatively, you can use the \code{OPENBLAS\_NUM\_THREADS}
environment variable to control the number of threads used by
OpenBLAS, and the \code{OMP\_NUM\_THREADS} environment variable to
control the number used in Jacobian calculations.


\section{Tips for the best performance}
\label{sec:tips}
\begin{itemize}
\item If you are working with single-threaded code, or in a
  multi-threaded program but with only one thread using a Stack
  object, then you can get slightly faster code by compiling all of
  your code with \code{-DADEPT\_STACK\_THREAD\_UNSAFE}. This uses a
  standard (i.e. non-thread-local) global variable to point to the
  currently active stack object, which is slightly faster to access.
\item If you compile with the \code{-g} option to store debugging
  symbols, your object files and executable will be much larger
  because every mathematical statement in the file will have the name
  of its associated templated type stored in the file, and these names
  can be long. Once you have debugged your code, you may wish to omit
  debugging symbols from production versions of the executable, or
  reduce the level of detail with \code{-g1} (on the GNU C++
  compiler).  There is typically no performance penalty associated
  with including debugging symbols.
\item A high compiler optimization setting is recommended to inline
  the function calls associated with mathematical expressions.  On the
  GNU C++ compiler, the \code{-O3 -march=native} setting is
  recommended.
\item On Intel architectures, \Adept\ will vectorize expressions if
  they satisfy a number of requirements: (1) they contain only
  elementary mathematical operators (including the functions
  \code{sqrt}, \code{max} and \code{min}), (2) the arrays in the
  expression are either all of type \code{float} or all of type
  \code{double}, (3) all the arrays in the expression must have their
  final dimension increasing in memory with no stride, and (4) none of
  the arrays are active. If AVX is enabled (\code{-mavx}) then four
  \code{double}s or eight \code{float}s will be operated on at once,
  otherwise if SSE2 is enabled (\code{-msse2}) then two \code{double}s
  or four \code{float}s will be operated on at once.
\item By default the Jacobian functions are compiled to process a
  strip of rows or columns of the Jacobian matrix at once. The optimum
  width of the strip depends on your platform, and you may wish to
  change it. To make the Jacobian functions process \textit{n} rows or
  columns at once, recompile the \Adept\ library with
  \code{-DADEPT\_MULTIPASS\_SIZE=}\textit{n}.
\item If you suspect memory usage is a problem, you may investigate
  the memory used by \Adept\ by simply sending your \code{Stack} object to a
  stream, e.g. ``\code{std::cout \textless\textless\ stack}''. You may
  also use the \code{memory()} member function, which returns the
  total number of bytes used. Further details of similar functions is
  given in section \ref{sec:stack}.
\end{itemize}

\section{Exceptions thrown by the \Adept\ library}
\label{sec:exceptions}
Some functions in the \Adept\ library can throw exceptions, and the
exceptions that can be thrown are typically derived from either
\code{adept::autodiff\_exception} or
\code{adept::array\_exception}. These classes are derived from
\code{adept::exception}, which is itself derived from
\code{std::exception}. Most indicate an error in the users code,
usually associated with calling \Adept\ functions in the wrong order.

An overly comprehensive exception-catching implementation that takes
different actions depending on whether a specific \Adept\ exception,
an exception related to automatic differentiation, a general
\Adept\ exception, or a non-\Adept\ exception is thrown, could have
the following form:
%
\begin{lstlisting}
 try {
   adept::Stack stack;
   // ... Code using the Adept library goes here ...
 }
 catch (adept::stack_already_active& e) {
   // Catch a specific Adept exception
   std::cerr << "Error: " << e.what() << std::endl;
   // ... any further actions go here ...
 }
 catch (adept::autodiff_exception& e) {
   // Catch any Adept exception related to automatic differentiation not yet caught
   std::cerr << "Error: " << e.what() << std::endl;
   // ... any further actions go here ...
 }
 catch (adept::exception& e) {
   // Catch any other Adept exception not yet caught
   std::cerr << "Error: " << e.what() << std::endl;
   // ... any further actions go here ...
 }
 catch (...) {
   // Catch any exceptions not yet caught
   std::cerr << "An error occurred" << std::endl;
   // ... any further actions go here ...
 }
\end{lstlisting}
%
All exceptions implement the \code{what()} member function, which
returns a \code{const char*} containing an error message. 

\subsection{General exceptions}
The following exceptions are not specific to arrays or automatic
differentiation and inherit directly from \code{adept::exception}::
\begin{description}
\citem{feature\_not\_available} This exception is thrown by deprecated
functions, such as \code{Stack::start()}. It is also thrown by
functions that are not available because a certain library is not
being used, such as \code{inv} if \Adept\ was compiled without LAPACK
support, or matrix multiplciation via the `\code{**}' psudo-operator
if \Adept\ was compiled without BLAS support.
\end{description}

\subsection{Automatic-differentiation exceptions}
The following exceptions relate to automatic differentiation (the
functionality described in chapter \ref{chap:ad}), and all are in the
\code{adept} namespace:
\begin{description}
\citem{gradient\_out\_of\_range} This exception can be thrown by the
\code{adouble::get\_gradient} member function if the index to its
gradient is larger than the number of gradients stored.  This can
happen if the \code{adouble} object was created after the first
\code{adouble::set\_gradient} call since the last
\code{Stack::new\_recording} call. The first
\code{adouble::set\_gradient} call signals to the \Adept\ stack that
the main algorithm has completed and so memory can be allocated to
store the gradients ready for a forward or reverse pass through the
differential statements. If further \code{adouble} objects are created
then they may have a gradient index that is out of range of the memory
allocated.
%
\citem{gradients\_not\_initialized} This exception can be thrown by
functions that require the list of working gradients to have been
initialized (particularly the functions
\code{Stack::compute\_tangent\_linear} and
\code{Stack::compute\_adjoint}). This initialization occurs when
\code{adouble::set\_gradient} is called.
%
\citem{stack\_already\_active} This exception is thrown when an
attempt is made to make a particular \code{Stack} object ``active'',
but there already is an active stack in this thread. This can be
thrown by the \code{Stack} constructor or the \code{Stack::activate}
member function.
%
\citem{dependents\_or\_independents\_not\_identified} This exception
is thrown when an attempt is made to compute a Jacobian but the
independents and/or dependents have not been identified.
%
\citem{wrong\_gradient} This exception is thrown by the
\code{adouble::append\_derivative\_dependence} if the \code{adouble}
object that it is called from is not the same as that of the most
recent \code{adouble::add\_derivative\_dependence}. 
%
\citem{non\_finite\_gradient} This exception is thrown if the users
code is compiled with the preprocessor variable
\code{ADEPT\_TRACK\_NON\_FINITE\_GRADIENTS} defined, and a
mathematical operation is carried out for which the derivative is not
finite. This is useful to locate the source of non-finite derivatives
coming out of an algorithm.
\end{description}

\subsection{Array exceptions}
\label{sec:array_exceptions}
The following exceptions relate to arrays (the functionality described
in chapter \ref{chap:arrays}), and all are in the \code{adept}
namespace:
\begin{description}
\citem{size\_mismatch} A mathematical operation taking two arguments
has been applied to array expressions that are not of the same
size. The same exception is thrown if an array expression is applied
to an array of a different size.
\citem{inner\_dimension\_mismatch} Matrix multiplication has been
attempted with arrays whose inner dimensions don't agree.
\citem{empty\_array} An empty array has been used in an operation when
a non-empty array is required; for example, if an attempt is made to
link an array to an empty array (see section \ref{sec:array} for more
information on linking).
\citem{invalid\_dimension} Attempt to create an array with a negative
dimension.
\citem{index\_out\_of\_bounds} An element or range of elements has
been requested from an array but one of the indices provided is out of
range; for a dimension of length $n$, the index is not in the range
$0$ to $n-1$. Note that bounds checking is only applied if the
preprocessor variable \code{ADEPT\_BOUNDS\_CHECKING} is defined.
%\citem{invalid\_lvalue}
\citem{invalid\_operation} An invalid operation has been performed
that can only be detected at run-time, for example, calling the
\code{diag\_submatrix} member function of a non-square rank-2
\code{Array}.
\citem{matrix\_ill\_conditioned} An attempt has been made to factorize
an ill-conditioned matrix (either via \code{solve} or \code{inv}).
\end{description}

\section{Configuring the behaviour of \Adept}
\label{sec:configuring}
The behaviour of the \Adept\ library can be changed by defining one or
more of the \Adept\ preprocessor variables. This can be done either by
editing the \code{adept/base.h} file and uncommenting the relevant
\code{\#define} lines, or by compiling your code with \code{-Dxxx}
compiler options (replacing \code{xxx} by the relevant preprocessor
variable. There are two types of preprocessor variable: the first
types only apply to the compilation of user code, while the second
types require the \Adept\ library to be recompiled.

\subsection{Modifications not requiring a library recompile}
\label{sec:configuring_no_lib}
The preprocessor variables that apply only to user code and do not
require the \Adept\ library to be recompiled are as follows:
\begin{description}
\citem{ADEPT\_STACK\_THREAD\_UNSAFE} If this variable is defined, the
currently active stack is stored as a global variable but is not
defined to be ``thread-local''. This is slightly faster, but means
that you cannot use multi-threaded code with separate threads holding
their own active \code{Stack} object. Note that although defining this
variable does not require a library recompile, all source files that
make up a single executable must be compiled with this option (or all
not be).
%
\citem{ADEPT\_RECORDING\_PAUSABLE} This option enables an algorithm to
be run both with and without automatic differentiation from within the
same program via the functions \code{Stack::pause\_recording()} and
\code{Stack::continue\_recording()}.  Note that although defining this
variable does not require a library recompile, all source files that
make up a single executable must be compiled with this option (or all
not be). Further details on this option are provided in section
\ref{sec:pausable}.
%
\citem{ADEPT\_NO\_AUTOMATIC\_DIFFERENTIATION} This option turns off
automatic differentiation by treating \code{adouble} objects as
\code{double}. It is useful if you want to compile one source file
twice to produce versions with and without automatic
differentiation. Further details on this option are provided in
section \ref{sec:multipleobjects}.
%
\citem{ADEPT\_TRACK\_NON\_FINITE\_GRADIENTS} Often when an algorithm
is first converted to use an operator-overloading automatic
differentiation library, the gradients come out as Not-a-Number or
Infinity. The reason is often that the algorithm contains operations
for which the derivative is not finite (e.g.\ $\sqrt{a}$ for $a=0$),
or constructions where a non-finite value is produced but subsequently
made finite (e.g.\ $\exp(-1.0/a)$ for $a=0$). Usually the algorithm
can be recoded to avoid these problems, if the location of the
problematic operations can be identified. By defining this
preprocessor variable, a \code{non\_finite\_gradient} exception will
be thrown if any operation results in a non-finite derivative. Running
the program within a debugger (and ensuring that the exception is not
caught within the program) enables the offending line to be
identified.
%
\citem{ADEPT\_INITIAL\_STACK\_LENGTH} This preprocessor variable is
set to an integer, and is used as the default initial amount of memory
allocated for the recording, in terms of the number of statements and
operations.
%
\citem{ADEPT\_REMOVE\_NULL\_STATEMENTS} If many variables in your code
are likely to be zero then redundant operations will be added to the
list of differential statements. For example, the assignment
$a=b\times c$ with active variables $b$ and $c$ both being zero
results in the differential statement $\delta a=0\times\delta
b+0\times\delta c$. This preprocessor variable checks for zeros and
removes terms on the right-hand-side of differential statements if it
finds them. In this case it would put $\delta a=0$ on the stack
instead. This option slows down the recording stage, but speeds up the
subsequent use of the recorded stack for adjoint and Jacobian
calculations. The speed up of the latter is only likely to exceed the
slow down of the former if your code contains many zeros. For most
codes, this option causes a net slow down.
%
\citem{ADEPT\_COPY\_CONSTRUCTOR\_ONLY\_ON\_RETURN\_FROM\_FUNCTION} In
\Adept\ 1.1 this enabled a small but unsafe optimization. It now has
no effect.
%
\citem{ADEPT\_BOUNDS\_CHECKING} If this variable is defined, check
that all array indices are within the bounds of the array throwing an
\code{index\_out\_of\_bounds} exception if necessary.  If this
variable is not defined then these checks are not performed, which is
faster but means that attempts to access arrays out of bounds will
result either of corruption of other memory used by the process, or a
segmentation fault. 
\citem{ADEPT\_NO\_ALIAS\_CHECKING} This variable turns off alias
checking, which results in faster code, but may lead to unexpected
results if the right-hand-side of an array statement shares data with
the left-hand-side of the expression. If this is likely for a
particular statement then use the \code{eval} function, described in
section \ref{sec:bounds}.
\citem{ADEPT\_STORAGE\_THREAD\_SAFE} This variable ensures that
accesses to the reference counter in \code{Storage} objects are
atomic, enabling the \code{Array} and \code{SpecialMatrix} objects
that use them to be accessed safely in a multi-threaded
environment. Note that this may incur a performance penalty, and is
only available in C++11. See section \ref{sec:thread}.
\end{description}

\subsection{Modifications requiring a library recompile}
\label{sec:configuring_lib}
\noindent The preprocessor variables that require the \Adept\ library
to be recompiled are as follows. Note that if these variables are used
they must be the same when compiling both the library and the user
code. This is safest to implement by editing section 2 of the
\code{adept/base.h} header file.
\begin{description}
\citem{ADEPT\_REAL\_TYPE\_SIZE} If you want to compile \Adept\ to use
a precision other than double for the \code{Real} type, and hence for
automatic differentiation, then define this preprocessor variable to
be \code{4} (for \code{float}), \code{8} (for \code{double}) or
\code{16} (for \code{long double}). This will also change the default
floating-point type for arrays, including shortcuts such as
\code{Vector}, \code{Matrix}, \code{SymmMatrix}. Note that if you
specify \code{16} but your compiler cannot support it
(i.e.\ \code{sizeof(long double)==8}) then \Adept\ would produce
sub-optimal code so will fail to compile.
%
\citem{ADEPT\_STACK\_STORAGE\_STL} Use the C++ standard template
library \code{vector} or \code{valarray} classes for storing the
recording and the list of gradients, rather than dynamically allocated
arrays. In practice, this tends to slow down the code.
%
\citem{ADEPT\_MULTIPASS\_SIZE} This is set to an integer, invariably a
power of two, specifying the number of rows or columns of a Jacobian
that are calculated at once. The optimum value depends on the platform
and the capability of the compiler to optimize loops whose length is
known at compile time.
% 
\citem{ADEPT\_MULTIPASS\_SIZE\_ZERO\_CHECK} This is also set to an
integer; if it is greater than \codebf{ADEPT\_MULTIPASS\_SIZE}, then
the \code{Stack::jacobian\_reverse} function checks gradients are
non-zero before using them in a multiplication.
%
\citem{ADEPT\_THREAD\_LOCAL} This can be used to specify the way that
thread-local storage is declared by your compiler.  Thread-local
storage is used to ensure that the \Adept\ library is thread-safe. By
default this variable is not defined initially, and then later in
\code{adept/base.h} it is set to an appropriate value on your system:
\code{thread\_local} if you compile with the C++11 standard, otherwise
\code{\_\_declspec(thread)} on Microsoft Visual C++, an empty
declaration on Mac (since thread-local storage is not available on
many Mac platforms) and \code{\_\_thread} otherwise (appropriate for
at least the GNU, Intel, Sun and IBM compilers). To override the
default behaviour, define this variable yourself in
\code{adept/base.h}.
\end{description}

\section{Frequently asked questions}
\label{sec:faq}
\begin{description}
\item[Why are all the gradients coming out of the automatic
  differentiation zero?] You have almost certainly omitted or
  misplaced the call of the \code{adept::Stack} member function
  ``\code{new\_recording()}''. It should be placed \emph{after} the
  independent variables in the algorithm have been initialized, but
  before any subsequent calculations are performed on these
  variables. If it is omitted or placed before the point where the
  independent variables are initialized, the differential statements
  corresponding to this initialization (which are all of the form
  $\delta x=0$), will be placed in the list of differential statements
  and will unhelpfully set to zero all your gradients right at the
  start of a forward pass (resulting from a call to \code{forward()})
  or set them to zero right at the end of a reverse pass (resulting
  from a call to \code{reverse()}).
\item[Why are the gradients coming out of the automatic
  differentiation NaN or Inf (even though the value is correct)?] This
  can occur if the algorithm contains operations for which the
  derivative is not finite (e.g.\ $\sqrt{a}$ for $a=0$), or
  constructions where a non-finite value is produced but subsequently
  made finite (e.g.\ $\exp(-1.0/a)$ for $a=0$). Usually the algorithm
  can be recoded to avoid these problems, if the location of the
  problematic operations can be identified. The simplest way to locate
  the offending statement is to recompile your code with the \code{-g}
  option and the \code{ADEPT\_TRACK\_NON\_FINITE\_GRADIENTS}
  preprocessor variable set (see section
  \ref{sec:configuring_no_lib}). Run the program within a debugger and
  a \code{non\_finite\_gradient} exception will be thrown, which if
  not caught within the program will enable you to locate the line in
  your code where the problem originated.  You may need to turn
  optimizations off (compile with \code{-O0}) for the line
  identification to be accurate. Another approach is to add the
  following in a C++ source file:
\begin{lstlisting}
 #include <fenv.h>
 int _feenableexcept_status = feenableexcept(FE_INVALID|FE_DIVBYZERO|FE_OVERFLOW);
\end{lstlisting}
  This will cause a floating point exception to be thrown when a
  \code{NaN} or \code{Inf} is generated, which can again be located in
  a debugger.
\item[Why are the gradients coming out of the automatic
  differentiation wrong?] Before suspecting a bug in \Adept, note that
  round-off error can lead to incorrect gradients even in hand-coded
  differential code. Consider the following:
\begin{lstlisting}
 int main() {
   Stack stack;
   adouble a = 1.0e-26, b;
   stack.new_recording();
   b = sin(a) / a;
   b.set_gradient(1.0);
   stack.compute_adjoint();
   std::cout << "a=" << a << ", b=" << b << ", db/da=" << a.get_gradient() << "\n";
 }
\end{lstlisting}
  We know that near \code{a=0} we should have \code{b=1} and the
  gradient should be \code{0}.  But running the program above will
  give a gradient of \code{1.71799e+10}. If you hand-code the
  gradient, i.e.
\begin{lstlisting}
 double A = 1.0e-26;
 double dB_dA = cos(A)/A - sin(A) / (A*A);
\end{lstlisting}
  you will you will also get the wrong gradient.  You can see that the
  answer is the difference of two very large numbers and so subject to
  round-off error.  This example is therefore not a bug of \Adept, but
  a limitation of finite-precision machines.  To check this, try
  compiling your code using either the ADOL-C or CppAD automatic
  differentiation tools; I have always found these tools to give
  exactly the same gradient as \Adept. Unfortunately, round-off error
  can build up over many operations to give the wrong result, so there
  may not be a simple solution in your case.
\item[Can \Adept\ reuse a stored tape for multiple runs of the same
  algorithm but with different inputs?] No. \Adept\ does not store the
  full algorithm in its stack (as ADOL-C does in its tapes, for
  example), only the derivative information.  So from the stack alone
  you cannot rerun the function with different inputs.  However,
  rerunning the algorithm including recomputing the derivative
  information is fast using \Adept, and is still faster than libraries
  that store enough information in their tapes to enable a tape to be
  reused with different inputs.  It should be stressed that for any
  algorithm that includes different paths of execution (``if''
  statements) based on the values of the inputs, such a tape would
  need to be rerecorded anyway. This includes any algorithm containing
  a look-up table.
\item[Why does my code crash with a segmentation fault?] This means it
  is trying to access a memory address not belonging to your program,
  and the first thing to do is to run your program in a debugger to
  find out at what point in your code this occurs. If it is in the
  \code{adept::aReal} constructor (note that \code{aReal} is synonymous with
  \code{adouble}), then it is very likely that you have tried to
  initiate an \code{adept::adouble} object before initiating an
  \code{adept::Stack} object. As described in section
  \ref{sec:stack_setup}, there are good reasons why you need to
  initialize the \code{adept::Stack} object first.
\item[How can I interface \Adept\ with a matrix library such as
  Eigen?]  Unfortunately the use of expression templates in
  \Adept\ means that it does not work optimally (if it works at all)
  with third-party matrix libraries that use expression
  templates. This is the reason why Adept 2.0 combines array
  functionality with automatic differentiation in a single
  expression-template framework.
\item[Do you have plans to enable \Adept\ to produce Hessian
  matrices?]  Not in the near future; refining the array functionality
  is a higher priority at the moment.  However, if your objective
  function $J(\x)$ (also known as a cost function or penalty function)
  has a quadratic dependence on each of the elements of $\y(\x)$,
  where $\y$ is a nonlinear function of the independent variables
  $\x$, then the Hessian matrix $\nabla_\x^2 J$ can be computed from
  the Jacobian matrix ${\bf H}=\partial\y/\partial\x$. This is the
  essence of the Gauss-Newton and Levenberg-Marquardt
  algorithms. Consider the optimization problem of finding the
  parameters $\x$ of nonlinear model $\y(\x)$ that provides the
  closest match to a set of ``observations'' $\y^o$ in a least-squares
  sense.  For maximum generality we add constraints that penalize
  differences between $\x$ and a set of \emph{a~priori} values $\x^a$,
  as well as a regularization term.  In this case the objective
  function could be written as \def\myspace{~~}
\begin{equation}
J(\x) \myspace =\myspace \left[\y(\x)-\y^o\right]^\mathrm{T}{\bf
  R}^{-1}\left[\y(\x)-\y^o\right]
\myspace+\myspace\left[\x-\x^a\right]^\mathrm{T}{\bf
  B}^{-1}\left[\x-\x^a\right]
\myspace+\myspace\x^\mathrm{T}{\bf T}\x.\nonumber
\label{eq:objective}
\end{equation}
  Here, all vectors are treated as column vectors, ${\bf R}$ is the
  error covariance matrix of the observations, ${\bf B}$ is the error
  covariance matrix of the \emph{a~priori} values, and ${\bf T}$ is a
  Twomey-Tikhonov matrix that penalizes either spatial gradients or
  curvature in $\x$.  The Hessian matrix is then given by
\begin{equation}
\nabla_\x^2J \myspace=\myspace {\bf H}^\mathrm{T}{\bf
  R}^{-1}{\bf H}\nonumber
\myspace+\myspace {\bf B}^{-1} \myspace+\myspace {\bf T},
\label{eq:hessian}
\end{equation}
  which can be coded up using \Adept\ to compute ${\bf H}$. Each term
  on the right-hand-side of (\ref{eq:hessian}) has its corresponding
  term in (\ref{eq:objective}), so it is easy to work out what the
  Hessian would look like if only a subset of the terms in
  (\ref{eq:objective}) were present.
\item[Why doesn't the ternary operator work?] Some compilers will fail
  to compile the following function:
\begin{lstlisting}
 adept::adouble piecewise(adept::adouble x) {
   return x < 1.0 ? x*x : 2.0*x-1.0;
 }
\end{lstlisting}%
The reason is that these compilers require that the two possible
outcomes of the ternary operator have the same type, but due to the
use of expression templates, the types of these mathematical
expressions are actually different.  The ternary operator cannot be
overloaded to allow such arguments. The solution is to explicitly
convert the outcomes to \code{adouble}:
\begin{lstlisting}
 adept::adouble piecewise(adept::adouble x) {
   return x < 1.0 ? adept::adouble(x*x) : adept::adouble(2.0*x-1.0);
 }
\end{lstlisting}
\item[Why is my executable so huge?]  Probably you are including
  debugging symbols by compiling with the \code{-g} option. Expression
  templates need long strings to describe them, so this extra content
  can increase the size of object files and executables by a factor of
  ten.  This does not slow down execution, but for production code you
  may wish to compile without debugging symbols, or if you use the GNU
  compiler use instead the \code{-g1} option which stores a reduced
  amount of debugging information.
\end{description}

\section{Copyright and license for \Adept\ software}
\label{sec:license}
Versions 1.9 of \Adept\ and later are owned and copyrighted jointly by
the University of Reading and the European Centre for Medium Range
Weather Forecasts. The copyright to versions 1.1 and earlier is held
solely by the University of Reading.

Since version 1.1, the \Adept\ library is released under the Apache
License, Version 2.0, which is available at
\url{http://www.apache.org/licenses/LICENSE-2.0}.  In short, this
free-software license permits you to use the library for any purpose,
and to modify it and combine it with other software to form a larger
work.  If you choose, you may release the modified software in either
source code or object code form, so may use \Adept\ in both
open-source software and non-free proprietary software. However,
distributed versions must retain copyright notices and also distribute
both the information in the NOTICES file and a copy of the Apache
License.  Different license terms may be applied to your distributed
software, although they must include the conditions on redistribution
provided in the Apache License.  This is a just short summary; if in
doubt, consult the text of the license.

In addition to the legally binding terms of the license, it is
\emph{requested} that:
\begin{itemize}
\item You cite \cite{Hogan2014} in publications describing algorithms
  and software that make use of the \Adept\ library. While not not a
  condition of the license, this is good honest practice in science
  and engineering.
\item If you make modifications to the \Adept\ library that might be
  useful to others, you release your modifications under the terms of
  the Apache License, Version 2.0, so that they are available to
  others and could also be merged into a future official version of
  \Adept. If you do not state the license applied to your
  modifications then by default they will be under the terms of the
  Apache License. You will retain copyright of your modifications, but
  if your modifications are written in the course of employment then
  under almost all circumstances (including employment by a
  University) it is your employer who holds the copyright.  Therefore
  you should obtain permission from them to release your modifications
  under the Apache License.
\end{itemize}

Note that other source files in the \Adept\ package used for
demonstrating and benchmarking \Adept\ are released under the GNU
all-permissive license\footnote{The GNU all-permissive license reads:
  \emph{Copying and distribution of this file, with or without
    modification, are permitted in any medium without royalty provided
    the copyright notice and this notice are preserved.  This file is
    offered as-is, without any warranty.}}, which is specified at the
top of all files it applies to.

\Adept\ version 1.0 was released under the terms of the GNU General
Public License (GPL) and so could not be released as part of a larger
work unless the entire work was released under the conditions of the
GPL.  It is hoped that the switch to the Apache License will
facilitate wider use of \Adept.

\section*{Acknowledgments}
Adept 1.0 was developed by Robin Hogan at the University of Reading
with funding from European Space Agency contract
40001041528/11/NL/CT. Some of the modifications to produce version 1.1
were funded by a National Centre for Earth Observation Mission Support
grant (Natural Environment Research Council grant NE/H003894/1). Dr
Brian Tse is thanked for his work exploring different parallelization
strategies during this period. Subsequent development has been carried
out under employment at the European Centre for Medium Range Weather
Forecasts.

\begin{thebibliography}{00}
\markright{References}
\harvarditem{Bell}{2007}{Bell2007}Bell, B., 2007: CppAD: A package for C++
algorithmic differentiation. \url{http://www.coin-or.org/CppAD}
% 
\harvarditem{Liu and Nocedal}{1989}{Liu+1989}Liu, D. C., and Nocedal,
  J., 1989: On the limited memory method for large scale
optimization. \emph{Math.\ Programming B,} {\bf 45,} 503--528.
%
\harvarditem{Gay}{2005}{Gay2005}Gay, D. M., 2005: Semiautomatic
differentiation for efficient gradient computations.  In
\emph{Automatic Differentiation: Applications, Theory, and
  Implementations}, H. M. B\"ucker, G. F. Corliss, P.  Hovland,
U. Naumann and B. Norris (eds.), Springer, 147--158.
%
\harvarditem{Griewank et~al.}{1996}{Griewank+1996}Griewank, A.,
  Juedes, D., and Utke, J., 1996:  Algorithm 755: ADOL-C: a package for the
automatic differentiation of algorithms written in C/C++. \textit{ACM
  Trans.\ Math.\ Softw.,} \textbf{22,} 131--167.
\harvarditem{Hogan}{2014}{Hogan2014}Hogan, R. J., 2014: Fast reverse-mode
  automatic differentiation using expression templates in
  C++. \textit{ACM Trans.\ Math.\ Softw.,} \textbf{40,} 26:1-26:16.
\harvarditem{Veldhuizen}{1995}{Veldhuizen1995}Veldhuizen, T., 1995:
Expression templates. {\it C++ Report,} {\bf 7,} 26--31.
\end{thebibliography}

\end{document}
